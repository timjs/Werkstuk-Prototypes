% !TeX root = werkstuk.tex
% !TeX program = xelatex

\chapter{Case study}

In de case study maken we aannemelijk dat de natuurlijke semantiek daadwerkelijk de werking beschrijft die wij voor ogen hebben. Dit doen we door een volledige afleidingsboom te geven voor het volgende exemplarische programma:

\newCodeFragment
\begin{minipage}{.5\textwidth}
\codeLines{
  \codeLine{\LOCAL \id{f}}
  \codeLine{\id{f} = \FUN(\id{n}) \RETURNS \id{g}}
  \codeLine{\IN \LOCAL \id{g}}
  \codeLine{\IN \id{g} = \FUN(\;) \RETURNS \id{n}}
  \codeLine{\IN \IN \id{n} = \id{n} + 1}
}
\end{minipage}
\begin{minipage}{.5\textwidth}
\codeLines{
  \codeLine{\LOCAL \id{c}}
  \codeLine{\id{c} = \id{f}(5)}
  \codeLine{\id{c}(\;)}
  \codeLine{\LOCAL \id{i}}
  \codeLine{\id{i} = \id{c}(\;)}
}
\end{minipage}

Laat hierbij een speciale \emph{start}-toestand $\State{\phantom{A}}$ als volgt gedefinieerd zijn:
%
\begin{equation*}
  \State{\phantom{A}} \DEF \langle(\varnothing, \nil)\rangle, \langle\rangle
\end{equation*}
%
%\begin{gather}
%  b_{\ms(@0)}(i) = 7
%  \textbf{~~waarbij~~}
%  \vTerminates{
%    \LOCAL \id{f} \\
%    \id{f} = \FUN(\id{n}) \RETURNS \id{g} \\
%    \IN \LOCAL \id{g} \\
%    \IN \id{g} = \FUN(\;) \RETURNS \id{n} \\
%    \IN \IN \id{n} = \id{n} + 1 \\
%    \LOCAL \id{c} \\
%    \id{c} = \id{f}(5) \\
%    \id{c}(\;) \\
%    \LOCAL \id{i} \\
%    \id{i} = \id{c}(\;)
%  }
%    {\State{\phantom{A}}}
%    {@0, \nil}
%    {\ms, \mo}
%  \label{cA} \\
%  \Express{\id{y}.\id{a}}{\ms, \mo, @0, \nil} = 42
%  \textbf{~~waarbij~~}
%  \vTerminates{
%    \LOCAL \id{x} \\
%    \id{x} \OBJECT \\
%    \id{x}.\id{a} = 42 \\
%    \LOCAL \id{y} \\
%    \id{y} \OBJECT \\
%    \id{y} \CLONES \id{x}
%  }
%    {\State{\phantom{A}}}
%    {@0, \nil}
%    {\ms, \mo}
%  \label{cB}
%\end{gather}
%
%Dit soort bewijzen gaan niet al te makkelijk met een natuurlijke semantiek, maar het is buiten de focus van dit werkstuk om een bijpassend axiomatische semantiek te ontwikkelen.
%
%We zullen daarom een groot deel van het bewijs dat redelijk simpel van karakter is (herhaaldelijk [comp], [local] etc) weglaten, en vooral de essentiële stappen bewijzen.
%
%Het volgende voorbeeld geeft weer hoe men bij simpele afleidingen met [comp], [local] en [assign var] te werk gaat. Laat $\State{A}$ en $\State{B}$ als volgt gedefinieerd zijn:
%%
%\begin{align*}
%f &\DEF \big<\left[\SmallProgram{
%        \LOCAL \id{g} \\
%        \id{g} = \FUN(\;) \RETURNS \id{n} \\
%        \IN \id{n} = \id{n} + 1
%      }\right], \langle\id{n}\rangle, \id{g}, @0\big> \\
%\State{A} &\DEF \langle([\id{f} \mapsto \nil], \nil)\rangle, \langle\rangle \\
%\State{B} &\DEF \langle([\id{f} \mapsto f], \nil)\rangle, \langle\rangle
%\end{align*}
%%
%Het eerste fragment voor het bewijs voor (\ref{cA}) gaat dan als volgt:
%%
%\begin{equation*}
%\AxiomC{}
%\LeftLabel{[local]}
%\UnaryInfC{$
%  \Terminates{\LOCAL \id{f}}
%    {\State{\phantom{A}}}
%    {@0, \nil}
%    {\State{A}}
%$}
%\AxiomC{}
%\LeftLabel{[assign var]}
%\UnaryInfC{$
%  \vTerminates{
%    \id{f} = \FUN(\id{n}) \RETURNS \id{g} \\
%    \IN \LOCAL \id{g} \\
%    \IN \id{g} = \FUN(\;) \RETURNS \id{n} \\
%    \IN \IN \id{n} = \id{n} + 1
%  }
%    {\State{A}}
%    {@0, \nil}
%    {\State{B}}
%$}
%\LeftLabel{[comp]}
%\BinaryInfC{$
%  \vTerminates{
%    \LOCAL \id{f} \\
%    \id{f} = \FUN(\id{n}) \RETURNS \id{g} \\
%    \IN \LOCAL \id{g} \\
%    \IN \id{g} = \FUN(\;) \RETURNS \id{n} \\
%    \IN \IN \id{n} = \id{n} + 1
%  }
%    {\State{\phantom{A}}}
%    {@0, \nil}
%    {\State{B}}
%$}
%\DisplayProof{}
%\end{equation*}
%%
%Nu nemen we aan dat bewezen is dat:
%%
%\begin{equation*}
%  \vTerminates{
%    \LOCAL \id{f} \\
%    \id{f} = \FUN(\id{n}) \RETURNS \id{g} \\
%    \IN \LOCAL \id{g} \\
%    \IN \id{g} = \FUN(\;) \RETURNS \id{n} \\
%    \IN \IN \id{n} = \id{n} + 1 \\
%    \LOCAL \id{c}
%  }
%    {\State{\phantom{A}}}
%    {@0, \nil}
%    {\State{C}}
%  \textbf{~~waarbij~~}
%  \State{C} \DEF \langle(\left[\Bindings{
%    \id{f} & f \\
%    \id{c} & \nil
%  }\right], \nil)\rangle, \langle\rangle
%\end{equation*}
%%
%De volgende 

\renewcommand{\SmallProgramSize}{.75}


\def\fgABCDetc{\begin{align*}
f &\DEF \big<\left[\SmallProgram{
    \LOCAL \id{g} \\
    \id{g} = \FUN(\;) \RETURNS \id{n} \\
    \IN \id{n} = \id{n} + 1
  }\right], \langle\id{n}\rangle, \id{g}, @0\big> &
\State{D} &\DEF \langle(\left[\Bindings{
    \id{f} & f \\
    \id{c} & \nil
  }\right], \nil), ([\id{n} \mapsto 5], @0)\rangle, \langle\rangle &
\State{I} &\DEF \langle(\left[\Bindings{
    \id{f} & f \\
    \id{c} & g
  }\right], \nil), (\left[\Bindings{
    \id{n} & 6 \\
    \id{g} & g
  }\right], @0), (\varnothing, @1)\rangle, \langle\rangle \\
g &\DEF \big<[\id{n} = \id{n} + 1], \langle\rangle, \id{n}, @1\big> &
\State{E} &\DEF \langle(\left[\Bindings{
    \id{f} & f \\
    \id{c} & \nil
  }\right], \nil), (\left[\Bindings{
    \id{n} & 5 \\
    \id{g} & \nil
  }\right], @0)\rangle, \langle\rangle &
\State{J} &\DEF \langle(\left[\Bindings{
    \id{f} & f \\
    \id{c} & g \\
    \id{i} & \nil
  }\right], \nil), (\left[\Bindings{
    \id{n} & 6 \\
    \id{g} & g
  }\right], @0), (\varnothing, @1)\rangle, \langle\rangle \\
\State{A} &\DEF \langle([\id{f} \mapsto \nil], \nil)\rangle, \langle\rangle &
\State{F} &\DEF \langle(\left[\Bindings{
    \id{f} & f \\
    \id{c} & \nil
  }\right], \nil), (\left[\Bindings{
    \id{n} & 5 \\
    \id{g} & g
  }\right], @0)\rangle, \langle\rangle &
\State{K} &\DEF \langle(\left[\Bindings{
    \id{f} & f \\
    \id{c} & g \\
    \id{i} & \nil
  }\right], \nil), (\left[\Bindings{
    \id{n} & 6 \\
    \id{g} & g
  }\right], @0), (\varnothing, @1), (\varnothing, @1)\rangle, \langle\rangle \\
\State{B} &\DEF \langle([\id{f} \mapsto f], \nil)\rangle, \langle\rangle &
\State{G} &\DEF \langle(\left[\Bindings{
    \id{f} & f \\
    \id{c} & g
  }\right], \nil), (\left[\Bindings{
    \id{n} & 5 \\
    \id{g} & g
  }\right], @0)\rangle, \langle\rangle &
\State{L} &\DEF \langle(\left[\Bindings{
    \id{f} & f \\
    \id{c} & g \\
    \id{i} & \nil
  }\right], \nil), (\left[\Bindings{
    \id{n} & 7 \\
    \id{g} & g
  }\right], @0), (\varnothing, @1), (\varnothing, @1)\rangle, \langle\rangle \\
\State{C} &\DEF \langle(\left[\Bindings{
    \id{f} & f \\
    \id{c} & \nil
  }\right], \nil)\rangle, \langle\rangle &
\State{H} &\DEF \langle(\left[\Bindings{
    \id{f} & f \\
    \id{c} & g
  }\right], \nil), (\left[\Bindings{
    \id{n} & 5 \\
    \id{g} & g
  }\right], @0), (\varnothing, @1)\rangle, \langle\rangle &
\State{M} &\DEF \langle(\left[\Bindings{
    \id{f} & f \\
    \id{c} & g \\
    \id{i} & 7
  }\right], \nil), (\left[\Bindings{
    \id{n} & 7 \\
    \id{g} & g
  }\right], @0), (\varnothing, @1), (\varnothing, @1)\rangle, \langle\rangle
\end{align*}}

\begin{Landscape}
\begin{prooftree}

\AxiomC{}
\RuleLabel{[local]}
\UnaryInfC{$
  \Terminates{\LOCAL \id{f}}
    {\State{\phantom{A}}}
    {@0, \nil}
    {\State{A}}
$}
\AxiomC{}
\RuleLabel{[assign]}
\UnaryInfC{$
  \vTerminates{
    \id{f} = \FUN(\id{n}) \RETURNS \id{g} \\
    \IN \LOCAL \id{g} \\
    \IN \id{g} = \FUN(\;) \RETURNS \id{n} \\
    \IN \IN \id{n} = \id{n} + 1
  }
    {\State{A}}
    {@0, \nil}
    {\State{B}}
$}
\RuleLabel{[comp]}
\BinaryInfC{$
  \vTerminates{
    \LOCAL \id{f} \\
    \id{f} = \FUN(\id{n}) \RETURNS \id{g} \\
    \IN \LOCAL \id{g} \\
    \IN \id{g} = \FUN(\;) \RETURNS \id{n} \\
    \IN \IN \id{n} = \id{n} + 1
  }
    {\State{\phantom{A}}}
    {@0, \nil}
    {\State{B}}
$}
\AxiomC{}
\RuleLabel{[local]}
\UnaryInfC{$
  \Terminates{\LOCAL \id{c}}
    {\State{\phantom{B}}}
    {@0, \nil}
    {\State{C}}
$}
\RuleLabel{[comp]}
\BinaryInfC{$
  \vTerminates{
    \LOCAL \id{f} \\
    \id{f} = \FUN(\id{n}) \RETURNS \id{g} \\
    \IN \LOCAL \id{g} \\
    \IN \id{g} = \FUN(\;) \RETURNS \id{n} \\
    \IN \IN \id{n} = \id{n} + 1 \\
    \LOCAL \id{c}
  }
    {\State{\phantom{A}}}
    {@0, \nil}
    {\State{C}}
$}
\AxiomC{}
\RuleLabel{[local]}
\UnaryInfC{$
  \Terminates{\LOCAL \id{g}}
    {\State{D}}
    {@1, \nil}
    {\State{E}}
$}
\AxiomC{}
\RuleLabel{[assign]}
\UnaryInfC{$
  \vTerminates{
    \id{g} = \FUN(\;) \RETURNS \id{n} \\
    \IN \id{n} = \id{n} + 1
  }
    {\State{E}}
    {@1, \nil}
    {\State{F}}
$}
\RuleLabel{[comp]}
\BinaryInfC{$
  \vTerminates{
    \LOCAL \id{g} \\
    \id{g} = \FUN(\;) \RETURNS \id{n} \\
    \IN \id{n} = \id{n} + 1
  }
    {\State{D}}
    {@1, \nil}
    {\State{F}}
$}
\RuleLabel{[call]}
\UnaryInfC{$
  \Terminates{\id{c} = \id{f}(5)}
    {\State{C}}
    {@0, \nil}
    {\State{G}}
$}
\RuleLabel{[comp]}
\BinaryInfC{$
  \vTerminates{
    \LOCAL \id{f} \\
    \id{f} = \FUN(\id{n}) \RETURNS \id{g} \\
    \IN \LOCAL \id{g} \\
    \IN \id{g} = \FUN(\;) \RETURNS \id{n} \\
    \IN \IN \id{n} = \id{n} + 1 \\
    \LOCAL \id{c} \\
    \id{c} = \id{f}(5)
  }
    {\State{\phantom{A}}}
    {@0, \nil}
    {\State{G}}
$}
\dottedLine
\UnaryInfC{\hspace*{.8\textwidth}}
\noLine
\UnaryInfC{(afgekapt, zie volgend blad)}
\end{prooftree}
\null
\vfill
\hrule
\fgABCDetc
\clearpage
\begin{prooftree}
\AxiomC{(vervolg afleiding, zie vorig blad)}
\dottedLine
\UnaryInfC{$
  \vTerminates{
    \LOCAL \id{f} \\
    \id{f} = \FUN(\id{n}) \RETURNS \id{g} \\
    \IN \LOCAL \id{g} \\
    \IN \id{g} = \FUN(\;) \RETURNS \id{n} \\
    \IN \IN \id{n} = \id{n} + 1 \\
    \LOCAL \id{c} \\
    \id{c} = \id{f}(5)
  }
    {\State{\phantom{A}}}
    {@0, \nil}
    {\State{G}}
$}
\AxiomC{}
\RuleLabel{[assign]}
\UnaryInfC{$
  \Terminates{\id{n} = \id{n} + 1}
    {\State{H}}
    {@2, \nil}
    {\State{I}}
$}
\RuleLabel{[call]}
\UnaryInfC{$
  \Terminates{\id{c}()}
    {\State{G}}
    {@0, \nil}
    {\State{I}}
$}
\RuleLabel{[comp]}
\BinaryInfC{$
  \vTerminates{
    \LOCAL \id{f} \\
    \id{f} = \FUN(\id{n}) \RETURNS \id{g} \\
    \IN \LOCAL \id{g} \\
    \IN \id{g} = \FUN(\;) \RETURNS \id{n} \\
    \IN \IN \id{n} = \id{n} + 1 \\
    \LOCAL \id{c} \\
    \id{c} = \id{f}(5) \\
    \id{c}()
  }
    {\State{\phantom{A}}}
    {@0, \nil}
    {\State{I}}
$}
\AxiomC{}
\RuleLabel{[local]}
\UnaryInfC{$
  \Terminates{\LOCAL \id{i}}
    {\State{I}}
    {@0, \nil}
    {\State{J}}
$}
\RuleLabel{[comp]}
\BinaryInfC{$
  \vTerminates{
    \LOCAL \id{f} \\
    \id{f} = \FUN(\id{n}) \RETURNS \id{g} \\
    \IN \LOCAL \id{g} \\
    \IN \id{g} = \FUN(\;) \RETURNS \id{n} \\
    \IN \IN \id{n} = \id{n} + 1 \\
    \LOCAL \id{c} \\
    \id{c} = \id{f}(5) \\
    \id{c}() \\
    \LOCAL \id{i}
  }
    {\State{\phantom{A}}}
    {@0, \nil}
    {\State{J}}
$}
\AxiomC{}
\RuleLabel{[assign]}
\UnaryInfC{$
  \Terminates{\id{n} = \id{n} + 1}
    {\State{K}}
    {@2, \nil}
    {\State{L}}
$}
\RuleLabel{[call]}
\UnaryInfC{$
  \Terminates{\id{i} = \id{c}()}
    {\State{J}}
    {@0, \nil}
    {\State{M}}
$}
\RuleLabel{[comp]}
\BinaryInfC{$
  \vTerminates{
    \LOCAL \id{f} \\
    \id{f} = \FUN(\id{n}) \RETURNS \id{g} \\
    \IN \LOCAL \id{g} \\
    \IN \id{g} = \FUN(\;) \RETURNS \id{n} \\
    \IN \IN \id{n} = \id{n} + 1 \\
    \LOCAL \id{c} \\
    \id{c} = \id{f}(5) \\
    \id{c}() \\
    \LOCAL \id{i} \\
    \id{i} = \id{c}()
  }
    {\State{\phantom{A}}}
    {@0, \nil}
    {\State{M}}
$}
\end{prooftree}
\null
\vfill
\hrule
\fgABCDetc
\end{Landscape}
