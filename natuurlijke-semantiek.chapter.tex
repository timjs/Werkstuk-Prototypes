
\chapter{Natuurlijke Semantiek}

\section{Expressies}

\section{Statements}

\subsection{Basis}

Laten we beginnen met de simpelste constructie in onze taal, het lege statement \SKIP. Deze heeft de vorm van een axioma.

\begin{NSAxiom}{skip}
  \begin{prooftree}
    \AxiomC{$
      \Config{\SKIP}{\ms, \mo, \sigma, \tau}
      \longrightarrow
      (\ms, \mo)
    $}
  \end{prooftree}
\end{NSAxiom}

Zoals we kunnen zien zijn onze uitspraken van de vorm
\begin{equation*}
  \Config{S}{\ms,\mo,\sigma,\tau} \longrightarrow (\ms',\mo').
\end{equation*}
Hiermee bedoelen we dat
\begin{equation*}
\big(\,(S,\ms,\mo,\sigma,\tau), (\ms',\mo')\,\big) \in (\longrightarrow),
\end{equation*}
waarbij $\longrightarrow$ de volgende signatuur heeft 
\begin{equation*}
  (\longrightarrow) \subseteq (\Stm \times \MMs \times \MMo \times \LLs \times \LLo) \times (\MMs \times \MMo).
\end{equation*}
Deze transitie werkt op een statement $S\in\Stm$ in een toestand $(\ms,\mo)\in(\MMs,\MMo)$ met als extra informatie de locatie van de huidige scope $\sigma\in\LLs$ en de locatie van het huidige \THIS-object $\tau\in\MMo$. Het resultaat is een nieuwe toestand in de vorm van de twee geheugens $(\ms',\mo')\in(\MMs,\MMo)$. \SKIP\ verandert niets aan de toestand zodat $(\ms',\mo')=(\ms,\mo)$.

Voor het samenstellen van statements hebben we een regel nodig.

\begin{NSAxiom}{comp}
  \begin{prooftree}
    \AxiomC{$
      \Config{S_1}{\ms, \mo, \sigma, \tau}
      \longrightarrow
      (\ms', \mo')
    $}
    \AxiomC{$
      \Config{S_2}{\ms', \mo', \sigma, \tau}
      \longrightarrow
      (\ms'', \mo'')
    $}
    \BinaryInfC{$
      \Config{S_1; S_2}{\ms, \mo, \sigma, \tau}
      \longrightarrow
      (\ms'', \mo'')
    $}
  \end{prooftree}
\end{NSAxiom}

In dit geval geven we aan dat, wanneer we een een compositie hebben van de statements $S_1$ en $S_2$, we eerst $S_1$ uitvoeren% T: Dit is niet het juiste woord, weten even geen betere...
en daarna $S_2$. Tijdens dit proces ontstaan nieuwe toestanden, waar we natuurlijk rekening mee moeten houden. De geheugens worden dan ook netjes doorgesluisd.

Voor de controlestructuur \IF\ hebben we twee regels nodig. De eerste is voor het geval dat de \BExpr\ evalueert in \T, dan moet namelijk het statement van het \THEN-deel worden uitgevoerd. Wanneer de \BExpr\ evalueert in \F\ moet het \ELSE-deel worden uitgevoerd.
Er moet dus aan een extra voorwaarde worden voldaan om deze regels toe te mogen passen. Dit zal vaker voorkomen bij de komende deductieregels. We noteren deze extra voorwaarden onder de regel of het axioma.
%Zoals vele axioma's en deductieregels heeft dit axioma een aantal voorwaarden waaraan voldaan moet worden. Deze staan eronder genoteerd, elk op een regel.  

\begin{NSAxiom}{if$^\T$}
  \begin{prooftree}
    \AxiomC{$
      \Config{S_1}{\ms, \mo, \sigma, \tau}
      \longrightarrow
      (\ms', \mo')
    $}
    \UnaryInfC{$
      \Config{\IF b \THEN  S_1 \ELSE S_2 }{\ms, \mo, \sigma, \tau}
      \longrightarrow
      (\ms', \mo')
    $}
  \end{prooftree}
  \begin{NSConditions}
    \Cond{$ \Surr{b}^\text{B}_{\ms, \mo,\sigma,\tau} = \T $}
  \end{NSConditions}
\end{NSAxiom}

en:

\begin{NSAxiom}{if$^\F$}
  \begin{prooftree}
    \AxiomC{$
      \Config{S_2}{\ms, \mo, \sigma, \tau}
      \longrightarrow
      (\ms', \mo')
    $}
    \UnaryInfC{$
      \Config{\IF b \THEN S_1 \ELSE S_2 }{\ms, \mo, \sigma, \tau}
      \longrightarrow
      (\ms', \mo')
    $}
  \end{prooftree}
  \begin{NSConditions}
    \Cond{$ \Surr{b}^\text{B}_{\ms, \mo,\sigma,\tau} = \F $}
  \end{NSConditions}
\end{NSAxiom}

Eenzelfde tactiek passen we toe bij een \WHILE-loop.

\begin{NSAxiom}{while$^\T$}
  \begin{prooftree}
    \AxiomC{$
      \Config{S_1}{\ms, \mo, \sigma, \tau}
      \longrightarrow
      (\ms',\mo')
    $}
    \AxiomC{$
      \Config{\WHILE b \DO S_1 }{\ms', \mo', \sigma, \tau}
      \longrightarrow
      (\ms'',\mo'')
    $}
    \BinaryInfC{$
      \Config{\WHILE b \DO S_1 }{\ms, \mo, \sigma, \tau}
      \longrightarrow
      (\ms'',\mo'')
    $}
  \end{prooftree}
  \begin{NSConditions}
    \Cond{$ \Surr{b}^\text{B}_{m,\sigma,\tau} = \T $}
  \end{NSConditions}
\end{NSAxiom}

en:

\begin{NSAxiom}{while$^\F$}
  \begin{prooftree}
    \AxiomC{$
      \Config{\WHILE b \DO S_1 }{\ms, \mo, \sigma, \tau}
      \longrightarrow
      (\ms, \mo)
    $}
  \end{prooftree}
  \begin{NSConditions}
    \Cond{$ \Surr{b}^\text{B}_{\ms,\mo,\sigma,\tau} = \F $}
  \end{NSConditions}
\end{NSAxiom}

\subsection{Variabelen}

We komen nu bij een interessanter deel van de taal, namelijk het \emph{declareren} van variabelen en het \emph{toekennen} van waarden. Dit gaat allemaal over scopes

\begin{NSAxiom}{declare}
  \begin{prooftree}
    \AxiomC{$
      \Config{\VAR i}{m, \sigma, \tau}
      \longrightarrow
      m'
    $}
  \end{prooftree}
  \begin{NSConditions}
    \Cond{$ m' = m \surr{ \sigma \mapsto \big(b_{m(\sigma)}[i \mapsto \BOT], p_{m(\sigma)}\big) } $}
  \end{NSConditions}
\end{NSAxiom}

\begin{NSAxiom}{assign$^\text{i}$}
  \begin{prooftree}
    \AxiomC{$
      \Config{i = e}{m, \sigma, \tau}
      \longrightarrow
      m'
    $}
  \end{prooftree}
  \begin{NSConditions}
    \Cond{$ \sigma_\text{def} = \textsc{Find}_m(\sigma, i) $}
    \Cond{$ \Surr{e}_{m, \sigma, \tau} = v $}
    \Cond{$ m'= m \surr{ \sigma_\text{def} \mapsto \big(b_{m(\sigma_\text{def})}[ i \mapsto v ], p_{m(\sigma_\text{def})}\big) } $}
  \end{NSConditions}
\end{NSAxiom}

\subsection{Objecten}

Bij attributen gaat dit net iets anders, we hebben de hulp nodig van extra functies en voorwaarden. Zo moeten we rekening houden met het doorlopen van een pad en het speciale geval onderscheiden dat het eerste deel van het pad \THIS\ is.

\begin{NSAxiom}{assign$^\text{slot}$}
  \begin{prooftree}
    \AxiomC{$
      \Config{i.s = e}{m, \sigma, \tau}
      \longrightarrow
      m'
    $}
  \end{prooftree}
  \begin{NSConditions}
    \Cond{$ \sigma_\text{def} = \textsc{Find}_m(\sigma, i) $}
    \Cond{$ b_{m(\sigma_\text{def})}(i) = \omega \in \LL $}
    \Cond{$ \textsc{Trav}_m(\omega, s) = (\omega', j) $}
    \Cond{$ \Surr{e}_{m, \sigma, \tau} = v $}
    \Cond{$ m'= m \surr{ \omega' \mapsto \big(b_{m(\omega')}[ j \mapsto v ], p_{m(\omega')}\big) } $}
  \end{NSConditions}
\end{NSAxiom}

\begin{NSAxiom}{assign$^\text{this}$}
  \begin{prooftree}
    \AxiomC{$
      \Config{\THIS.s = e}{m, \sigma, \tau}
      \longrightarrow
      m'
    $}
  \end{prooftree}
  \begin{NSConditions}
    \Cond{$ \textsc{Trav}_m(\tau, s) = (\omega, i) $}
    \Cond{$ \Surr{e}_{m, \sigma, \tau} = v $}
    \Cond{$ m'= m \surr{ \omega \mapsto \big(b_{m(\omega)}[ i \mapsto v ], p_{m(\omega)}\big) } $}
  \end{NSConditions}
\end{NSAxiom}

\begin{NSAxiom}{object}
  \begin{prooftree}
    \AxiomC{$
      \Config{i \OBJECT}{m, \sigma, \tau}
      \longrightarrow
      m''
    $}
  \end{prooftree}
  \begin{NSConditions}
    \Cond{$ \textsc{Find}_m(\sigma, i) = \sigma_\text{def} $}
    \Cond{$ m' = m \surr{ \sigma_\text{def} \mapsto \big(b_{m(\sigma')}[i\mapsto \omega], p_{m(\sigma')}\big) } $}
    \Cond{$ m'' = m' \surr{ \omega \mapsto \big(\varnothing, \BOT\big) } $}
  \end{NSConditions}
\end{NSAxiom}

\begin{NSAxiom}{clones}
  \begin{prooftree}
    \AxiomC{$
      \Config{i \CLONES j}{m, \sigma, \tau}
      \longrightarrow
      m'
    $}
  \end{prooftree}
  \begin{NSConditions}
    \Cond{$ \Surr{i}_{m,\sigma,\tau} = \omega_i \in \LL $}
    \Cond{$ \Surr{j}_{m,\sigma,\tau} = \omega_j \in \LL $}
    \Cond{$ m' = m \surr{ \omega_i \mapsto \big(b_{m(\omega_i)}, \omega_j\big) } $}
  \end{NSConditions}
\end{NSAxiom}

\subsection{Functies}

\begin{NSAxiom}{call}
  \begin{prooftree}
    \AxiomC{$
      \Config{S_{\!f}}{m', \sigma_{\!f\text{new}}, \omega'}
      \longrightarrow
      m''
    $}
    \UnaryInfC{$
      \Config{i.s(e^*)}{m,\sigma,\tau}
      \longrightarrow
      m''
    $}
  \end{prooftree}
  \begin{NSConditions}
    \Cond{$ \sigma_\text{def} = \textsc{Find}_m(\sigma, i) $}
    \Cond{$ b_{m(\sigma_\text{def})}(i) = \omega \in \LL $}
    \Cond{$ \textsc{Trav}_n(\omega, s) = (\omega', j) $}
    \Cond{$ (S_{\!f}, I_{\!f}, i_{\!f}, \sigma_{\!f\text{def}}) = f = b_{m(\omega')}(j) $}
    \Cond{$ \sigma_{\!f\text{new}} = \textsc{Next}_\text{scope}(m) $}
    \Cond{$ m' = m\surr{ \sigma_{\!f\text{new}} \mapsto \big(\Surr{e^*}^*_{m,\sigma,\tau}(I_{\!f}), \sigma_{\!f\text{def}}\big) } $}
  \end{NSConditions}
\end{NSAxiom}

\section*{Extra}

%Deze teksten zijn vooral bedoeld als ``tekstvlees'' (lorem ipsum's). We zullen axioma's en deductieregels introduceren waarmee we de relatie $(\longrightarrow)$ definiëren, die de volgende signatuur heeft:

%$$ (\longrightarrow) \subseteq (\Stm \times \MM \times \LL \times \LL) \times \MM $$

%Wanneer we een uitspraak doen van de vorm:

%$$ \Config{S}{m,\sigma,\tau} \longrightarrow m' $$

%..dan bedoelen we daarmee dat:

%$$ \big(\,(S,m,\sigma,\tau), m'\,\big) \in (\longrightarrow) $$

Deze uitspraak moet je lezen als: ``In de toestand met geheugen $m$, scope $\sigma$ en \emph{this} object $\tau$, termineert het statement $S$, waarbij het resultaat-geheugen $m'$ is.''

Een van deze axioma's [object], heeft betrekking tot de productieregel in de grammatica die de $\OBJECT$ ``literal'' introduceert.

Wanneer bij een dergelijke opsomming van voorwaarden een nieuwe variabele wordt geïntroduceerd zoals hierboven, met de volgende vorm: $\textbf{desda } \square = \theta \dots$; dan moet deze gelezen worden als: $\textbf{desda } \exists_\theta \surr{ \square = \theta \dots }$.

% vim: syn=latex spell spl=nl cole=1 cocu=nv
