\chapter{Notatie en terminologie}

In dit hoofdstuk behandelen we zowel een aantal gebruikelijke wiskundige concepten, als een aantal specifieke notaties en begrippen die in dit werkstuk vaak zullen terugkeren. Gezien het aard van het onderwerp zullen we bijvoorbeeld vaak over \emph{eindige} functies en verzamelingen spreken.

\section{Functies}

In dit werkstuk identificeren we een functie met zijn grafiek, dit wil zeggen dat een functie $f : X \to Y$ wordt gedefiniëerd door de verzameling paren $(x, y) \in X \times Y$ waarvoor we beweren dat $f(x) = y$. Uiteraard voldoet zo'n verzameling $f \subseteq X \times Y$ aan de voorwaarde dat
\begin{equation*}
  \neg \exists_{x \in X,\, y_1 \in Y,\, y_2 \in Y} \surr{ (x, y_1) \in f \land (x, y_2) \in f \land y_1 \neq y_2 }
\end{equation*}
De reden voor deze aanpak is zeker niet fundamenteel: het is gewoonweg handig om $\varnothing$ te schrijven voor een nieuwe, ``lege'', partiële functie.

\section{Partiële functies}

Vrijwel alle functies die we in dit werkstuk behandelen zijn partiële functies. Wanneer een partiële functie $f : X \to Y$ niet gedefiniëerd is op een zeker punt $x$ (dus $\neg \exists_{y \in Y} \surr { (x, y) \in f }$) schrijven we $f(x) = \bot$. Wanneer het omgekeerde het geval is, schrijven we $f(x) \neq \bot$, of kortweg $f(x) = y$ voor de gecombineerde uitspraak dat $f$ wél gedefiniëerd is op $x$ én dat $(x, y) \in f$.

Voor een willekeurige term $\phi = \dots f(x)\dots$, waarbij $f$ een partiële functie is die niet gedefiniëerd is op punt $x$, geldt ook dat $\phi = \bot$. Op deze manier is het niet nodig om te schrijven: ``als $f(x) = \bot$, dan $z = \bot$; anders als $f(x) \neq \bot$, dan $z = \phi$''. Deze ``verkorte schrijfwijze'' stelt ons in staat om op een elegantie manier functie definities op te schrijven. Een voorbeeld:
\begin{align*}
  f : \NN \times \NN &\to \NN \\
  f(0, 0) &= 1 \\
  f(n, m + 1) &= f(n, m)
\end{align*}
In dit voorbeeld geldt voor alle $m \in \NN$ dat $f(0, m) = 1$, en voor alle $n \in \NN \setminus \{0\}$ is $f(n, m)$ niet gedefiniëerd.

\section{Eindige functies, eindige verzamelingen}

De reden dat de meeste behandelde functies partiëel zijn is omdat meeste onderdelen van ons semantisch model eindig van karakter zijn. Functies worden vaak gebruikt om ``een verzameling variabelen die een bepaalde waarde bevatten'' te representeren, bijvoorbeeld de gedefiniëerde variabelen in een zekere scope. Het zou ongewoon zijn om in een programmeertaal gebruik te maken van scopes waarin oneindig veel variabelen kunnen bestaan.

Wanneer we het over een eindige functie $f$ hebben, bedoelen we daarmee dat het domein van die functie een eindige verzameling is. Dit kan worden uitgedrukt door te zeggen dat er een zeker getal $N \in \NN$ bestaat, zó dat $\underline{N}$ gelijkmachtig is aan het domein van $f$.
\begin{multline*}
  f : X \to Y \text{ is ``eindig'' } \DEF \\
  \exists_{N \in \NN} \surr{ \text{er bestaat een bijectie van $\underline{N}$ naar } \{ x \in X \mid f(x) \neq \bot \} }
\end{multline*}
Hierbij is $\underline{N}$ gedefinieerd als de verzameling van de eerste $N$ natuurlijke getallen, ofwel $\{n \in \NN \mid n < N\}$.

We schrijven $\FiniteFunctions{Y}{X}$ voor de verzameling functies $\{f : X \to Y \mid f \text{ ``eindig''} \}$.

\section{Functies uitbreiden}
\label{sec:uitbreiden}

Het is vaak handig om een functie op een later tijdstip \emph{uit te breiden}. Hiermee bedoelen we dat de functie ongewijzigd blijft op alle punten $(x, y) \in f$, behalve één specifiek punt $x_1$ dat we willen koppelen aan $y_1$ zodat geldt:
%
\begin{equation*}
  f(x_1) = y_1
\end{equation*}
%
Dit geven we aan in met de notatie:
%
\begin{equation*}
  f\surr{ x_1 \mapsto y_1 }
\end{equation*}
%
Wanneer we meerdere aanpassingen willen maken, bijvoorbeeld $x_2$ koppelen aan $y_2$ en $x_3$ aan $y_3$ kan dat met bovenstaande notatie als volgt
%
\begin{equation*}
  \big(f\surr{ x_2 \mapsto y_2 }\big)\surr{ x_3 \mapsto y_3 }
\end{equation*}
%
\dots hetgeen we afkorten tot:
%
\begin{equation*}
  f\surr{ x_2 \mapsto y_2, x_3 \mapsto y_3 }
\end{equation*}
%
Heel precies gezegd, als $f : X \to Y$ een functie is, $x_n \in X$ en $y_n \in Y$, dan:
%
\begin{multline*}
  f\surr{ x_n \mapsto y_n } \DEF f' \;\mathbf{desda} \\
  \forall_{x \in X,\; y \in Y} \surr{ (x, y) \in f' \Leftrightarrow \big( (x, y) \in f \land x \neq x_n \big) \lor (x, y) = (x_n, y_n) }
\end{multline*}

\section{Lijsten}

We zullen meermaals in ons werkstuk gebruik maken van willekeurig grote, maar altijd eindige, \emph{lijsten} van elementen uit een zekere verzameling. Deze lijsten worden gerepresenteerd door eindige (partiële) functies $t : \NN \to X$ (met $X$ de verzameling waaruit we de elementen van de lijst nemen), waaraan nog een paar extra voorwaarden worden gesteld. De verzameling van alle lijsten op een zekere verzameling $X$, genoteerd $X_{\langle\rangle}$, is als volgt gedefiniëerd:
\begin{equation*}
X_{\langle\rangle} \DEF \big\{\, t : \NN \to X \mid \exists_{N \in \NN} \surr{ \forall_{n < N}[t(n) \neq \bot] \land \forall_{n\ge N}[t(n) = \bot] } \,\big\}
\end{equation*}
We schrijven $\langle\rangle$, maar ook wel $\varnothing$ aangezien het gewoon een lege functie is zoals beschreven in het voorgaande, voor de lege lijst. Deze is natuurlijk altijd hetzelfde, ongeacht welke invulling wordt gekozen voor $X$.

Als $t$ een zekere lijst is, dan zeggen we dat hij van \emph{grootte} $N = \min\{n\in\NN \mid t(n) = \bot\}$ is.
We schrijven $\langle x_0, x_1, \dots, x_{N-1}\rangle$ voor de lijst $t$ van grootte $N$ waarvoor geldt dat $\forall_{n < N}[t(n) = x_n]$.
Als $t$ een lijst is uit $X_{\langle\rangle}$, en $x$ een element van $X$, dan schrijven we $t:x$, de toevoeging van $x$ aan de lijst $t$, voor de lijst $t' = t[N \mapsto x]$, waarbij $N$ de grootte van $t$ is.

\section{Notationele conventies}

Terwille van leesbaarheid en elegantie houden we een aantal gebruikelijke notationele conventies aan.

Veel wiskundige formules zijn van de vorm $t_1\; R\; t_2$, waarbij $R$ een zeker predikaat is (mogelijk $=$), en $t_1$ en $t_2$ termen. Dit soort formules zullen we wel vaker ``samenstellen'' tot formules als:
\begin{gather*}
  6 = 2 \cdot 3 > 2 \ge 42 - 40 \\
  f(x) = y \in Y
\end{gather*}
De intentie is enkel een elegante schrijfwijze te hanteren die makkelijk en intuïtief leest.
Als we bovenstaande formules uitschrijven krijgen we:
\begin{gather*}
  6 = 2 \cdot 3 \land 2 \cdot 3 > 2 \land 2 \ge 42 - 40 \\
  f(x) = y \land y \in Y
\end{gather*}

% vim: syn=latex spell spl=nl cole=1 cocu=nv
