\chapter{Notatie en terminologie}

In dit hoofdstuk behandelen we een aantal min of meer gebruikelijke wiskundige concepten, die in dit werkstuk vaak zullen terugkeren. Gezien het aard van het onderwerp zullen we bijvoorbeeld vaak over \emph{eindige} functies en verzamelingen willen spreken. Verder hanteren we een aantal specifieke, soms enigszins ``verkorte'', wiskundige schrijfwijzen ten dienste van leesbaarheid en elegantie.

\section{Functies}

In dit werkstuk identificeren we een functie met zijn grafiek, d.w.z.~een functie $f : X \to Y$ wordt gedefiniëerd door de verzameling paren $(x, y) \in X \times Y$ waarvoor we beweren dat $f(x) = y$. Uiteraard voldoet zo'n verzameling $f \subseteq X \times Y$ aan de voorwaarde dat: $$ \not \exists_{x \in X,\, y_1 \in Y,\, y_2 \in Y} \surr{ (x, y_1) \in f \land (x, y_2) \in f \land y_1 \neq y_2 ] } $$ De reden voor deze aanpak is zeker niet fundamenteel: het is gewoonweg handig om $\varnothing$ te schrijven voor een nieuwe, ``lege'', partiële functie.

\section{Partiële functies}

Vrijwel alle functies die we in dit werkstuk behandelen zijn partiële functies. Wanneer een partiële functie $f$ niet gedefiniëerd is op een zeker punt $x$, schrijven we $f(x) = \bot$. Wanneer het omgekeerde het geval is, schrijven we $f(x) \neq \bot$.

Voor een willekeurige term $\phi = \dots f(x)\dots$, waarbij $f$ een partiële functie is die niet gedefiniëerd is op punt $x$, geldt dat $\phi = \bot$. Op deze manier hoeven we niet iets omslachtigs te schrijven als: ``als $f(x) = \bot$, dan $z = \bot$; anders als $f(x) \neq \bot$, dan $z = \phi$''.

\section{Eindige functies, eindige verzamelingen}

De reden dat de meeste behandelde functies partiëel zijn is omdat meeste onderdelen van ons semantisch model eindig van karakter zijn. Functies worden vaak gebruikt om ``een verzameling variabelen die een bepaalde waarde bevatten'' te representeren, bijvoorbeeld de gedefiniëeerde variabelen in een zekere scope. En het zou ongewoon zijn om in een programmeertaal gebruik te maken van oneindige objecten, of scopes waarin oneindig veel variabelen bestaan.

Wanneer we het over een eindige functie $f$ hebben, bedoelen we daarmee dat het domein van die functie een eindige verzameling is. Dit kan worden uitgedrukt door te zeggen dat er een zeker getal $n \in \sN$ bestaat, zó dat $\underbar{n}$ (gedefiniëerd als: $\{m \in \sN \mid m \le n\}$) isomorf is aan het domein van $f$.

$$ f : X \to Y \text{ is ``eindig'' } \DEF \exists_{n \in \sN} \surr{ \underbar{n} \cong \{ x \in X \mid f(x) \neq \bot \} } $$

We schrijven $\FiniteFunctions{Y}{X}$ voor de verzameling functies $\{f : X \to Y \mid f \text{ ``eindig''} \}$.

\section{Tupels}

We zullen meermaals in ons werkstuk gebruik maken van willekeurig grote, maar altijd eindige, ``lijsten'' van elementen uit een zekere verzameling: \emph{tupels}. Deze tupels worden gerepresenteerd door eindige (partiële) functies $t : \sN \to X$, als X de verzameling element in kwestie is, waaraan nog een paar extra voorwaarden worden gesteld. De verzameling van alle tupels op een zekere verzameling $X$, genoteerd $X_{\langle\rangle}$, is als volgt gedefiniëerd:

$$ X_{\langle\rangle} \DEF \big\{\, t : \sN \to X \mid \exists_{N \in \sN} \big[ \forall_{n < N}[t(n) \neq \bot] \land \forall_{n\ge N}[t(n) = \bot] \big] \,\big\} $$

We schrijven $\langle\rangle$, maar ook wel $\varnothing$, voor de lege tupel (deze is natuurlijk hetzelfde voor elke waarden verzameling $X$).

We schrijven $\langle x_0, x_1, \dots, x_{N-1}\rangle$ voor de tupel $t$ waarvoor geldt: $\forall_{n < N}[t(n) = x_n]$, en: $\forall_{n \ge N}[t(n) = \bot]$.

Als $t$ een tupel is $\in X_{\langle\rangle}$, en $x$ een element van $X$, dan schrijven we $t:x$ voor de tupel $t' = t[N \mapsto x]$, waarbij $N = \min\{n\in\sN \mid t(n) = \bot\}$.

\section{Notationele conventies}

Terwille van leesbaarheid en elegantie houden we een aantal gebruikelijke notationele conventies aan.

\begin{itemize}
  \item Veel wiskundige formules zijn van de vorm $t_1\; R\; t_2$, waarbij $R$ een zeker predikaat is (mogelijk $=$), en $t_1$ en $t_2$ termen. Dit soort formules zullen we wel vaker ``samenstellen'' tot formules als:
  \begin{eqnarray}
    6 = 2 * 3 > 2 \ge 42 - 40 \\
    f(x) = y \in Y
  \end{eqnarray}
  De intentie is enkel een elegante schrijfwijze te hanteren die makkelijk en intuïtief leest. Zo zou een college analyse onverdraaglijk zijn als dergelijke verkortingen niet zouden worden gebruikt. Als we bovenstaande formules uitschrijven krijgen we:
  \begin{eqnarray}
    6 = 2 * 3 \land 2 * 3 > 2 \land 2 \ge 42 - 40 \\
    f(x) = y \land y \in Y
  \end{eqnarray}
  \item Bij het definiëren van verzamelingen gebruiken we vaak de volgende voor de hand liggende notatie: $$ \{a \in A \mid \phi \} $$ \dots i.p.v.~het omslachtigere: $$ \{a \mid a \in A \mid \phi\} $$
\end{itemize}
