
\chapter{Notatie en terminologie}

\section{Functies}

In dit werkstuk identificeren we een functie met zijn grafiek, d.w.z.~een functie $f : X \to Y$ is gelijk aan de verzameling paren $(x, y) \in X \times Y$ waarvoor geldt: $f(x) = y$.

\section{Tupels}

We zullen meermaals in ons werkstuk gebruik maken van willekeurig grote, maar altijd eindige, ``lijsten'' van elementen uit een zekere verzameling: \emph{tupels}. Deze tupels worden gerepresenteerd door eindige (partiële) functies $t : \sN \to X$, als X de verzameling element in kwestie is, waaraan nog een paar extra voorwaarden worden gesteld. De verzameling van alle tupels op een zekere verzameling $X$, genoteerd $X_{\langle\rangle}$, is als volgt gedefiniëerd:

$$ X_{\langle\rangle} \DEF \big\{\, t : \sN \to X \mid \exists_{N \in \sN} \big[ \forall_{n < N}[t(n) \neq \bot] \land \forall_{n\ge N}[t(n) = \bot] \big] \,\big\} $$

We schrijven $\langle\rangle$, maar ook wel $\varnothing$, voor de lege tupel (deze is natuurlijk hetzelfde voor elke waarden verzameling $X$).

We schrijven $\langle x_0, x_1, \dots, x_{N-1}\rangle$ voor de tupel $t$ waarvoor geldt: $\forall_{n < N}[t(n) = x_n]$, en: $\forall_{n \ge N}[t(n) = \bot]$.

Als $t$ een tupel is $\in X_{\langle\rangle}$, en $x$ een element van $X$, dan schrijven we $t:x$ voor de tupel $t' = t[N \mapsto x]$, waarbij $N = \min\{n\in\sN \mid t(n) = \bot\}$.

\section{Beschouwing semantisch model}

We definieren in dit werkstuk een natuurlijke semantiek, d.w.z.~een ?-ste orde logica, met axioma's en deductieregels, en een bijbehorende structuur waarin deze zich afspeelt.

Deze structuur, die we ook wel het \emph{semantisch model} zullen noemen, heeft onderstaand opgesomde elementen. Deze worden verderop precies gedefinieerd, onderstaande opsomming geeft slechts een algemeen beeld.

\begin{description}
	\item[$\mathbb{M}$]\hfill\\ De verzameling mogelijke \emph{geheugens}, welke ook wel als \emph{eindtoestanden} worden geinterpreteerd.
	\item[$(\mathit{Stm} \times \mathbb{M} \times \mathbb{L} \times \mathbb{L})$]\hfill\\ De verzameling \emph{toestanden}, ook wel \emph{configuraties}.
	\item[$(\longrightarrow)$]\hfill\\ Een tweeplaatsig predikaat welke als eerste argument een element uit de verzameling van toestanden neemt, en als tweede argument een element uit de verzameling van eindtoestanden $(\mathbb{M}\dots)$. De uitspraak $(S, m, \sigma, \tau) \longrightarrow m'$ moet worden geinterpreteerd worden als:
	\begin{quote} ``Het programma $S$, met geheugen $m$, in scope $\sigma$ en met als $\mathbf{this}$ object $\tau$, resulteert in eindtoestand $m'$, mits $S$ \emph{correct} is''. \end{quote}
\end{description}

\section{Notationele conventies}

Terwille van elegantie houden we een aantal gebruikelijke notationele conventies aan:

\begin{enumerate}
	\item Voor elke twee willekeurige tweestemmige predikaten $\mathsf{S}$ en $\mathsf{T}$ (mogelijk ook $=$), en drie willekeurige elementen $a$, $b$ en $c$, definieren we de afkorting: $$a \operatorname{\mathsf{S}} b \operatorname{\mathsf{T}} c \buildrel{\mathrm{def}}\over{=} a \operatorname{\mathsf{S}} b \land b \operatorname{\mathsf{T}} c$$ in het geval dat deze bewering correct getypeerd is.
	\item Op eenzelfde manier definieren we ook de volgende afkorting: $$ \{a \in A \mid \phi \} \buildrel{\mathrm{def}}\over{=} \{a \mid a \in A \mid \phi\}$$
\end{enumerate}

[...]
