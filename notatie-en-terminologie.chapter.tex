\chapter{Notatie en terminologie}
\label{chp:notatie}

In dit hoofdstuk behandelen we zowel een aantal gebruikelijke wiskundige concepten, als een aantal specifieke notaties en begrippen die in dit werkstuk vaak zullen terugkeren. Gezien de aard van het onderwerp zullen we bijvoorbeeld vaak over \emph{eindige}, \emph{discrete} functies en verzamelingen spreken.

\section{Functies}
\label{sec:afbeeldingen}

We identificeren een functie met zijn grafiek, dit wil zeggen dat een functie $f : X \to Y$ wordt gedefinieerd door de verzameling paren $(x, y) \in X \times Y$ waarvoor we beweren dat $f(x) = y$. Daarmee is een functie $f : X \to Y$ een speciaal geval van een relatie $f \subseteq X \times Y$, waarbij aan de \emph{functionele} voorwaarde wordt voldaan dat:
%
\begin{equation*}
  \neg \exists_{x \in X,\; y_1,y_2 \in Y} \left[ (x, y_1) \in f \land (x, y_2) \in f \land y_1 \neq y_2 \right]
\end{equation*}

\section{Partiële functies}
\label{sec:partielefuncties}

Vrijwel alle functies die we in dit werkstuk behandelen zijn partiële functies. Wanneer een partiële functie $f : X \to Y$ niet gedefinieerd is op een zeker punt $x$, dus $\neg \exists_{y \in Y} [ (x, y) \in f ]$, schrijven we $f(x) = \bot$ of soms ook $f \uparrow x$. Wanneer het omgekeerde het geval is, schrijven we $f(x) \neq \bot$, of soms $f \downarrow x$, of kortweg $f(x) = y$ voor de gecombineerde uitspraak dat $f$ wél gedefinieerd is op $x$ én dat $(x, y) \in f$.

Voor een willekeurige term $\phi = \dots f(x)\dots$, waarbij $f$ een partiële functie is die niet gedefinieerd is op punt $x$, geldt ook dat $\phi = \bot$. Op deze manier is het niet nodig om te schrijven: “als $f(x) = \bot$, dan $z = \bot$; anders als $f(x) \neq \bot$, dan $z = \phi$”. Deze “verkorte schrijfwijze” stelt ons in staat om op een elegante manier functies te definiëren. Een voorbeeld:
\begin{align*}
  f : \NN \times \NN &\to \NN \\
  f(0, 0) &= 1 \\
  f(n, m + 1) &= f(n, m)
\end{align*}
In dit voorbeeld geldt voor alle $m \in \NN$ dat $f(0, m) = 1$, en voor alle $n \in \NN \setminus \{0\}$ is $f(n, m)$ niet gedefinieerd.

\section{Eindige verzamelingen, eindige functies}
\label{sec:eindigefuncties}

De reden dat de meeste behandelde functies partieel zijn, is omdat meeste onderdelen van ons semantisch model eindig van karakter zijn. Functies worden vaak gebruikt om “een verzameling variabelen die een bepaalde waarde bevatten” te representeren, bijvoorbeeld de gedefinieerde variabelen in een zeker bereik. Het zou ongewoon zijn om in een programmeertaal gebruik te maken van bereiken waarin oneindig veel variabelen kunnen bestaan.

Een verzameling heet \emph{eindig} als deze voor zekere $n \in \NN$ gelijkmachtig is aan $\{n \in \NN \mid n < N\}$.
Een functie $f: X \to Y$ heet \emph{eindig} wanneer de verzameling $\{x \in X \mid f \downarrow x\}$ eindig is.
We schrijven in dit werkstuk $Y^X$ voor de verzameling functies $\{f : X \to Y \mid f \text{ eindig} \}$, dit in tegenstelling tot de gebruikelijke definitie waarin $Y^X$ alle functies $f : X \to Y$ bevat.

\section{Functie uitbreidingen}
\label{sec:uitbreiden}

Het is vaak handig om een functie $f : X \to Y$ op een later tijdstip \emph{uit te breiden} tot een nieuwe functie $f'$, waarbij $f'$ ongewijzigd blijft ten opzichte van $f$ op alle punten $(x, y) \in f$, behalve één specifiek punt $x_1 \in X$ dat we koppelen aan $y_1 \in Y$ zodat geldt: $f'(x_1) = y_1$. De definitie van $f'$, gebruik makend van $f$, noteren we als volgt:
%
\begin{equation*}
  f' \DEF f \left[ x_1 \mapsto y_1 \right]
\end{equation*}
%
Waarbij $f'$ aan de volgende eis voldoet:
%
\begin{equation*}
  \forall_{x \in X,\; y \in Y} \left[ (x, y) \in f' \Leftrightarrow \left( (x, y) \in f \land x \neq x_n \right) \lor (x, y) = (x_1, y_1) \right]
\end{equation*}
%
Meerdere uitbreidingen verkorten we op voor de hand liggende wijze:
%
\begin{equation*}
  f \left[ x_1 \mapsto y_1,\; x_2 \mapsto y_2 \dots x_n \mapsto y_n \right]
  \DEF
  f \left[ x_1 \mapsto y_1 \right]
    \left[ x_2 \mapsto y_2 \right]\dots
    \left[ x_n \mapsto y_n \right]
\end{equation*}

\section{Lijsten}
\label{sec:lijsten}

We zullen meermaals in ons werkstuk gebruik maken van willekeurig grote, maar altijd eindige, \emph{lijsten} van elementen uit een zekere verzameling. Deze lijsten worden gerepresenteerd door eindige partiële functies $t : \{(n, n) \in \NN^2\} \to X$ (met $X$ de verzameling waaruit we de elementen van de lijst nemen), waaraan nog een paar extra voorwaarden worden gesteld. We zullen ook wel $\II$ schrijven i.p.v.~$\{(n, n) \in \NN^2\}$, om aan te geven dat het gaat om de indices van lijsten. Merk op dat we als indices geen gebruik maken van $\NN$. Later in het werkstuk gaan we er vanuit dat we elementen uit $\II$ kunnen onderscheiden van elementen uit $\ZZ$. De verzameling van alle lijsten op een zekere verzameling $X$, genoteerd $X_{\langle\rangle}$, is als volgt gedefinieerd:
\begin{equation*}
X_{\langle\rangle} \DEF \big\{\, t : \II \to X \big\vert \exists_{N \in \NN} \surr{ \forall_{n < N}[t \downarrow (n, n)] \land \forall_{n\ge N}[t \uparrow (n, n)] } \,\big\}
\end{equation*}
We schrijven $\langle\rangle$, maar ook wel $\varnothing$ aangezien het gewoon een lege functie is zoals beschreven in het voorgaande, voor de lege lijst. Deze is natuurlijk altijd hetzelfde, ongeacht welke invulling wordt gekozen voor $X$.

Een lijst $t$ heeft \emph{grootte}
%
\begin{equation*}
  |t| = N = \min\{n\in\NN \mid t \uparrow (n, n)\}.
\end{equation*}
%
We schrijven $\langle x_0, x_1, \dots, x_{N-1}\rangle$ voor de lijst $t$ van grootte $N$ met $\forall_{n < N}[t(n, n) = x_n]$.
Als $t$ een lijst is uit $X_{\langle\rangle}$, en $x$ een element van $X$, dan schrijven we $t:x$, de toevoeging van $x$ achteraan de lijst $t$, voor de lijst $t' = t[(N, N) \mapsto x]$, waarbij $N = |t|$.

\section{Notationele conventies}
\label{sec:conventies}

Terwille van leesbaarheid en elegantie houden we een aantal gebruikelijke notationele conventies aan.

Veel wiskundige formules zijn van de vorm $t_1\; R\; t_2$, waarbij $R$ een zeker predikaat is (mogelijk $=$), en $t_1$ en $t_2$ termen. Dit soort formules zullen we wel vaker “samenstellen” tot formules als:
\begin{gather*}
  6 = 2 \cdot 3 > 2 \ge 42 - 40 \\
  f(x) = y \in Y
\end{gather*}
De intentie is enkel een elegante schrijfwijze te hanteren die makkelijk en intuïtief leest.
Als we bovenstaande formules uitschrijven krijgen we:
\begin{gather*}
  6 = 2 \cdot 3 \land 2 \cdot 3 > 2 \land 2 \ge 42 - 40 \\
  f(x) = y \land y \in Y
\end{gather*}

% vim: syn=latex spell spl=nl cole=1
