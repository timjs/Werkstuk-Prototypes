\chapter{Semantisch model}

\section{Bindingen}\label{sec:bindingen}

Aan de basis van ons model ligt het concept van een \emph{binding}. Een binding is een toekenning van een \emph{waarde} aan een variabele (een element uit de syntactische verzameling \Id). Bindingen zijn bijvoorbeeld van belang om de gedefiniëerd variabelen binnen een bereik vast te leggen, of de attributen van een bepaald object. Een \emph{groep bindingen} is een eindige functie $b : \Id \to \VV$. De verzameling van alle groepen van bindingen definiëren we dus als
\begin{equation*}
  \BB \DEF \FiniteFunctions{\VV}{\Id}
\end{equation*}
We komen later terug op wat de waarden $\VV$ precies zijn in §\ref{sec:waarden}. Voor nu is het voldoende om te weten dat in ieder geval de natuurlijke getallen $\NN$ deel uitmaken van $\VV$.

Bindingen komen veelvuldig terug in ons model. In bereiken worden \emph{variabelen} gedeclareerd en aan waarden gekoppeld. Bij objecten zijn het de \emph{attributen} die waarden krijgen toegekend.

\section{Bereik en omliggende bereiken}\label{sec:bereiken}

In sectie \ref{sec:voorbeelden} is informeel gebleken dat bereiken conceptueel goed te zien zijn als een boomstructuur. Stel we evalueren een variabele \id{x} in bereik $s$:

\newCodeFragment
\codeLines{
  \codeLine{\id{x}}[We bevinden ons in een zeker bereik $s$.]
}

dan kunnen we dit als volgt uitleggen. Eerst zoeken we \id{x} op in de bindingen groep $b_s$, behorende bij bereik $s$.
%
\begin{equation*}
  b_s(x).
\end{equation*}
%
Vervolgens moet er gevalsonderscheiding worden gedaan voor de volgende situaties:

\begin{enumerate}
  \item \id{x} is gedefiniëerd in $b_s$, dus gebruiken we de gevonden waarde.
  \item \id{x} is niet gedefiniëerd in $b_s$, dus moeten we \id{x} opzoeken in de omliggend bereik.
\end{enumerate}

Hieruit blijkt dat we voor een willekeurig bereik niet alleen zijn eigen bindingen moeten bijhouden, maar ook een verwijzing naar zijn \emph{omliggend bereik}. Een bereik $s$ definiëren we daarom als een paar $(b, \pi)$, met $b$ de bindingen en $\pi$ een \emph{verwijzing} naar de omliggend bereik (ook wel \emph{parent}).
% "outer bereik" ;)

We moeten benadrukken dat $\pi$ een \emph{verwijzing} is, en niet een \emph{kopie} van de bindingen groep van de omliggend bereik. Stel dat we het programma in code fragment~\ref{exa:lexical} uitvoeren. Op het moment dat we $f()$ aanroepen in regel~\ref{exa:lexical:eerste} willen we dat $x$ daarna evalueert naar de waarde $2$. Evenzo moet $x$ na regel~\ref{exa:lexical:tweede} evalueren naar de waarde $4$. Het bereik $s_f$ van functie $f$ heeft een eigen binding $b_f$ die gedurende de executie van het programma leeg is, $x$ is namelijk niet gedeclareerd als een \LOCAL variabele. De omliggend bereik $\pi_f$ van functie $f$ verwijst naar bereik $s$, zodat de variabele $x$ uiteindelijk wel gevonden wordt.

\begin{NoBreak}
  \newCodeFragment[exa:lexical][Lexicaal bereik: opslaan en terugvinden van variabelen]
  \codeFragmentCaption
  \codeLines{
    \codeLine{\VAR \id{x}}
    \codeLine{\id{x} = 1}
    \codeLine{\VAR \id{f}}
    \codeLine[exa:lexical:def]{\id{f} = \FUN()}[Introductie nieuw bereik]
    \codeLine{\IN \id{x} = 2 \times \id{x}}
    \codeLine{}[Einde nieuw bereik]
    \codeLine[exa:lexical:eerste]{\id{f}()}[$\id{x} = 2$]
    \codeLine[exa:lexical:tweede]{\id{f}()}[$\id{x} = 4$]
  }
\end{NoBreak}

Stel nu dat we geen verwijzing in het bereik opslaan, maar een kopie van de omliggende bindingen. Op het moment dat we $f$ definiëren in regel~\ref{exa:lexical:def} is bereik $s_f$ een paar $(b_f, p_f)$ met $b_f \in \BB$. Net als hierboven zijn de eigen bindingen $b_f$ leeg. De binding $p_f$ bevat een functie onder naam \id{f} en de waarde $1$ onder naam \id{x}. Wanneer we \id{x} aanpassen door de aanroep in regel~\ref{exa:lexical:eerste} wordt dit doorgevoerd in de binding $p_f$ maar, omdat dit een kopie is, niet in de binding $b_s$ van de omliggend bereik $s$. We moeten dus wel een verwijzing opslaan willen we het gevraagde gedrag krijgen. Daarnaast wordt het met kopieën erg lastig om een boomstructuur te creëren zodat we een variabele nog hogerop kunnen opzoeken.

Een bereik $s$ is dus een element uit de verzameling
%
\begin{equation*}
  \SS \DEF \BB \times (\LLs \cup \{\nil\}).
\end{equation*}
%
Hierbij zijn $\BB$ de bindinggroepen zoals besproken in §\ref{sec:bindingen}. $\LLs$ zijn locaties van bereiken. Op het begrip ``locatie'' komen wij nog terug in §\ref{sec:locaties}. We moeten er wel rekening mee houden dat er een soort ``ultiem'' omliggend bereik is. Het kan dus zijn dat een bereik geen parent heeft. Voor dat geval introduceren we het unieke element $\nil$. De beoogde interpretatie van een bereik van de vorm $(b, \nil)$ is dan dat het geen omliggend bereik heeft.

\section{Objecten en prototype overerving}
\label{sec:sem-objecten}

In §\ref{sec:taal-prototypen} hebben we een beeld gekregen van prototype overerving. Net als bereik en omliggend bereik, worden objecten en hun prototypen het best gemodelleerd met een boomstructuur. Geheel in lijn met bereiken is een object een paar met daarin zijn eigen bindingen $b$ en een verwijzing naar zijn prototype $\pi$. Natuurlijk kan een object ook geen prototype hebben. Dit geven we weer aan met $\nil$. Een object $o$ is dus een element uit
%
\begin{equation*}
  \OO \DEF \BB \times (\LLo \cup \{\nil\}).
\end{equation*}
%
Hierbij is $\BB$ weer de verzamelingen bindinggroepen uit §\ref{sec:bindingen} en $\LLo$ zijn ditmaal locaties van objecten.

Hoewel het semantische model voor objecten en hun prototypen vrijwel identiek is aan het semantische model voor bereiken, hebben ze een andere betekenis. Om deze gelijkheid te benadrukken gebruiken we voor beide modellen een aantal zelfde letters, en om de ongelijkheid te benadrukken subscripten we de locatie verzamelingen ($\LLo$ en $\LLs$).

\section{Functies}
\label{sec:sem-functies}

Van een functie moeten een aantal dingen bekend zijn:
\begin{description}
  \item[Code] De code (ook wel \emph{body}) van de functie is natuurlijk simpelweg een \Stm.
  %
  \item[Argumenten] Deze worden gerepresenteerd door een lijst van elementen uit \Id. In het programma zelf is deze lijst een \MaybeIds. We definiëren daarom de triviale functie $\textsc{Ids} : \MaybeIds \to \Id_{\langle\rangle}$ op de volgende manier:
  \begin{align*}
    \textsc{Ids}(\varepsilon) &= \langle\rangle \\
    \textsc{Ids}(i) &= \{(0, i)\} \\
    \textsc{Ids}(i^+, i) &= \textsc{Ids}(i^+) : i
  \end{align*}
  \dots waarbij we $i^+$ als meta-variabele gebruiken voor een element uit \Ids.
  %
  \item[Return variabele] Mogelijk gebruikt de functie een return variabele, dit wordt gerepresenteerd door een element uit $(\Id \cup \{\nil\})$, waarbij $\nil$ de betekenis draagt dat er geen sprake is van een return variabele.
  %
  \item[Bereik van definitie] Het is belangrijk om van een functie zijn bereik van originele definitie te weten. Dit is het kern-idee van lexicaal bereik. Het kan best zo zijn dat een zekere functie $f$ op een zekere lexicale plek in het programma is gedefiniëerd, en vervolgens door toedoen van het programma in een andere context wordt uitgevoerd. Maar het is belangrijk dat het lexicale bereik van de originele definitie nog bekend is, om vrije variabelen in de code van de functie op te kunnen zoeken in omgevende bereiken. Dit bereik wordt gerepresenteerd door de locatie van het bereik, een element uit \LLs.
\end{description}

Een functie is daarom simpelweg een viertupel met bovenstaande informatie. De verzameling functies is gedefiniëerd als:
\begin{equation*}
  \FF \DEF \Stm \times \Id_{\langle\rangle} \times (\Id \cup \{\nil\}) \times \LL
\end{equation*}

\section{Waarden en data}
\label{sec:waarden}
% K: Ik vond de vertelvorm waarin jij uitlegde wat waarden en locaties en primitieve waarden etc zijn toch enigszins "kinderachtig". Ik geloof dat een intuitieve terminologie [ik ben nu gesettled voor <data> en <waarden>, maar dat kunnen we veranderen..] en een wat meer "beschouwende" (i.p.v. uitleggende) tekst stukken beter werkt. Ik heb daarom deze sectie vrijwel helemaal herschreven, ik hoop dat je dat niet erg vindt! En overigens moet jij natuurlijk deze tekst weer ff onder handen nemen, want ik ben er zeker niet al te zeker over ;) Zo heb ik het over dat een programma "spreekt" over data: ??

In voorgaande paragrafen spraken we over waarden $\VV$ en locaties $\LL$ ($\LLs$ voor bereiken, $\LLo$ voor objecten). We hebben de exacte definities hiervan in het midden gelaten. Wat wel duidelijk is dat een waarde $v \in \VV$ is iets wat we toekennen aan een \Id\ met behulp van een binding.

In §\ref{subsec:taal-types} is uitgelegd hoe de taal met verschillende typen data omgaat. In onze taal komen namelijk drie typen \emph{data} voor: getallen, functies en objecten. Er is echter een verschil in de manier waarop wij ze behandelen in het semantisch model. Deze verschillende aanpak ik bekend aan iedereen die een keer object-geörienteerd geprogrammeerd heeft.
Zoals meeste object-geörienteerde programmeertalen maakt ook onze taal verschil tussen twee soorten data: primitieve waarden en objecten. Primitieve waarden vatten we op als de ``atomen'' waaruit programma's en objecten worden opgebouwd. Denk aan natuurlijke getallen of woorden.

\begin{NoBreak}
  \newCodeFragment[exa:prititive-values-numbers][Toekenning van primitieve waarden: Getallen]
  \codeFragmentCaption
  \codeLines{
    \codeLine{\VAR \id{x}}
    \codeLine{\id{x} = 5}[de primitieve waarde ``5'' wordt aan variabele \id{x} toegekend]
    \codeLine{}
    \codeLine{\VAR \id{y}}
    \codeLine{\id{y} = \id{x}}[nu bevat ook \id{y} de primitieve waarde ``5'']
    \codeLine{}
    \codeLine{\id{x} = 6}[\id{x} bevat ``6'', maar \id{y} bevat nog steeds ``5'']
  }
\end{NoBreak}

In onze taal zijn ook functies primitieve waarden. Zoals in \ref{sec:sem-functies} wordt beschreven, beschouwen we een functie als een ``atomair'' element met de volgende informatie:
\begin{itemize}
  \item parameters
  \item mogelijkerwijs een return variabele
  \item code
  \item bereik van definitie
\end{itemize}
Ondanks dat dit een hoop informatie is, is het slechts een bouwsteen van complexe structuren. Het volgende voorbeeld illustreert deze gedachte:

\begin{NoBreak}
  \newCodeFragment[exa:prititive-values-functions][Toekenning van primitieve waarden: Functies]
  \codeFragmentCaption
  \codeLines{
    \codeLine{\VAR \id{x}}
    \codeLine{\id{x} = \FUN (\;) \{ \SKIP \}}
    \codeLine{}
    \codeLine{\VAR \id{y}}
    \codeLine{\id{y} = \id{x}}
    \codeLine{}
    \codeLine{\id{x} = \FUN(\id{z}) \{ \id{z} = 5 \}}[\id{x} bevat de nieuwe functie met één argument,]
    \codeLine{}[maar \id{y} bevat nog steeds de oude \SKIP-functie]
  }
\end{NoBreak}

Objecten worden anders behandeld. Het idee van een object is dat dit een zekere ``entiteit'' voorstelt. Deze entiteit kan een nabootsing zijn van een objecten uit de reële wereld, maar kan ook iets abstracts zijn. Net als objecten uit de reële wereld, kan een object in een programma veranderen in de loop der tijd maar nog wel ``hetzelfde ding zijn als tevoren''. Het volgende code fragment illustreert dit idee:

\begin{NoBreak}
  \newCodeFragment[exa:prititive-values-functions][Toekenning van primitieve waarden: Functies]
  \codeFragmentCaption
  \codeLines{
    \codeLine{\VAR \id{x}}
    \codeLine{\id{x} \OBJECT}
    \codeLine{\id{x}.\id{n} = 5}
    \codeLine{}
    \codeLine{\VAR \id{y}}
    \codeLine{\id{y} = \id{x}}[\id{y} ``verwijst'' nu naar hetzelfde object als \id{x}]
    \codeLine{}
    \codeLine{\id{x}.\id{n} = 6}[omdat \id{x} en \id{y} naar hetzelfde object verwijzen: $\id{y}.\id{n} = 6$]
  }
\end{NoBreak}

In zekere zin zijn \id{x} en \id{y} immaterieel, omdat ze op deze manier ``vervangbaar'' zijn. Dit in tegenstelling to \id{n}. Het zijn dan ook verschillende dingen: \id{x} en \id{y} zijn variabelen; \id{n} is een attribuut van een zeker object waar zowel \id{x} en \id{y} naar verwijzen.
Het verwijzen van een variabele naar een object wordt in onze semantiek bewerkstelligd door het object ergens op te slaan, en de locatie ervan als waarde aan de variabele toe te kennen.

Om samen te vatten:
\begin{description}
  \item[Data] De wereld van data waar een programma mee om gaat bevat getallen, functies en objecten.
  \item[Waarden] Een waarde is wat er in ons semantisch model aan een variabele wordt toegekend, dit met behulp van een binding. Primitieve waarden (getallen, functies) zijn zelf waarden. Om in een programma over objecten te spreken, wordt de locatie van een object ook als waarde beschouwd.
\end{description}

De definitie van de verzameling van alle waarden (locaties van objecten, getallen en functies) is dus als volgt:

\begin{equation*}
  \VV \DEF \LLo \cup \NN \cup \FF
\end{equation*}

\section{Locaties en geheugen}
\label{sec:locaties}

Uit de redenaties over referenties volgt dat we een plaats moeten hebben om alle objecten op te slaan. Dit noemen we het \emph{geheugen voor objecten} $m_o$. Voor een gegeven locatie $\omega\in\LLo$ kunnen we het bijbehorende object in een geheugen $m_o$ opzoeken:
%
\begin{equation*}
  m_o(\omega).
\end{equation*}
%
Een geheugen is dus een functie van locaties naar objecten. De verzameling van alle geheugens definiëren we als
%
\begin{equation*}
  \MMo \DEF \FiniteFunctions{\OO}{\LLo}
\end{equation*}

\section{Expressiesemantiek}

[nog doen]

\section{Hulpfuncties}

We introduceren nu een aantal hulpfuncties om de logica van bepaalde delen van het semantische model te omvatten en te beschrijven.
Zoals eerder beschreven, moet de definitie van een variabele eerst gezocht worden in het huidige bereik, en daarna in omliggende bereiken. Deze procedure manifesteert zich in de hulpfunctie $\textsc{Find}_\text{scope} : \MMs \times \LLs \times \Id \to \LLs$, die we als volgt definiëren:
%
\begin{equation*}
  \textsc{Find}_\text{scope}(\ms, \sigma, i) = \begin{cases}
    \sigma & \textbf{als } b_{\ms(\sigma)}(i) \neq \bot \\
    \bot & \textbf{als } p_{\ms(\sigma)} = \nil \\
    \textsc{Find}_\text{scope}(\ms, p_{\ms(\sigma)}, i) & \textbf{anders}
  \end{cases}
\end{equation*}
%
Voor het doorzoeken van de prototype hiërarchie naar een attribuut hebben we een hulpfunctie $\textsc{Find}_\text{proto} : \MMo \times \LLo \times \Id \to \LLo$, die identiek is aan $\textsc{Find}_\text{scope}$, behalve dat deze werkt met de relevante object-gerelateerde verzamelingen.
%
\begin{equation*}
  \textsc{Find}_\text{proto}(\mo, \omega, i) = \begin{cases}
    \omega & \textbf{als } b_{\mo(\omega)}(i) \neq \bot \\
    \bot & \textbf{als } p_{\mo(\omega)} = \nil \\
    \textsc{Find}_\text{proto}(\mo, p_{\mo(\omega)}, i) & \textbf{anders}
  \end{cases}
\end{equation*}
%
Er is ook een hulpfunctie nodig om een pad te doorlopen. De hulpfunctie $\textsc{Trav} : \MMo \times \LLo \times \Path \to \LLo \times \Id$ levert de locatie van het laatste object op en de laatste \Id\ van het pad (wat mogelijk naar niet noodzakelijk een attribuut is van dat laatste object), gegeven het objectgeheugen, een locatie voor het ``begin'' object, en een pad.
%
\begin{equation*}
  \textsc{Trav}(\mo, \omega, p) = \begin{cases}
    (\omega, p) & \textbf{als } p \in \Id \\
    \textsc{Trav}(\mo, \omega', p') & \textbf{als } p = i.p' \textbf{ en } \textsc{Find}_\text{proto}(\mo, \omega, i) = \omega' \in \LLo \\
    \bot & \textbf{anders}
  \end{cases}
\end{equation*}
%

%\section*{Overzicht}
%
%\begin{align*}
%  \tag*{locaties van scopes en objecten, DONE}
%  \LL &\DEF \{ (n, n) \in \NN^2 \} \\
%  \tag*{functies}
%  \FF &\DEF \Stm \times \Id_{\langle\rangle} \times (\Id \cup \{\nil\}) \times \LL \\
%  \tag*{waarden, DONE}
%  \VV &\DEF \LL \cup \NN \cup \FF \\
%  \tag*{bindinggroepen, DONE}
%  \BB &\DEF \FiniteFunctions{\VV}{\Id} \\
%  \tag*{objecten, DONE}
%  \OO &\DEF \BB \times (\LL \cup \{\nil\}) \\
%  \tag*{scopes, DONE}
%  \SS &\DEF \BB \times (\LL \cup \{\nil\}) \\
%\end{align*}

% vim: syn=latex spell spl=nl cole=1 cocu=nv
