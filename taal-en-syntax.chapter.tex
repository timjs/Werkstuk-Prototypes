
\chapter{Taal en syntaxis}

In dit hoofdstuk presenteren we de taal waarvoor we een natuurlijk semantiek construeren. De taal maakt gebruikt van prototype overerving en lexicaal bereik. Eerst beschouwen we een aantal voorbeeldprogramma's, om zo informeel de te formaliseren taal te karakteriseren. Daarna geven we een rigoureuze definitie met behulp van een \BNF\ grammatica.

De structuur van de productieregels van deze grammatica worden in latere hoofdstukken gebruikt om axioma's en deductieregels op te stellen. Daarmee heeft de grammatica in zekere zin een dubbele functie.
Het is belangrijk om te vermelden dat het er hierbij niet gaat om een taal te maken die er “mooi” uit ziet. De taal bevat enkel onderdelen die essentiëel zijn voor het modelleren en formaliseren van lexicaal bereik en prototype overerving. Om dezelfde reden moet de syntaxis van de taal worden beschouwd als een mogelijke representatie van een \emph{abstract syntax tree} van een “echte” programmeertaal. We zullen dan ook, waar mogelijk, puntkomma's en haakjes weglaten. Het gebruik van regeleindes en inspringen van blokken geeft, naar ons idee, binder belemmering bij het lezen van een programma.

\section{Voorbeeldprogramma's}
\label{sec:voorbeelden}

Elk voorbeeldprogramma en zijn toelichtingen worden als volgt gepresenteerd:

\begin{NoBreak}
\newCodeFragment[exa:template][Het eerste voorbeeldprogramma]
\codeFragmentCaption
\begin{codelines}
  \codeLine{\LOCAL \id{f}}[$\id{f}$ moet eerst worden gedefiniëerd]
  \codeLine{\id{f} = \FUN(\id{i}) \RETURNS \id{n}}
  \codeLine{\IN \LOCAL \id{n}}
  \codeLine{\IN \id{n} = 2 \times (\id{i} + 5)}
  \codeLine{}[$\id{x}$ bestaat niet in deze scope]
  \codeLine{\LOCAL \id{x}}[\id{x} heeft nog geen waarde, maar is wel gedefiniëerd]
  \codeLine{\id{x} = \id{f}(42)}[\id{x} heeft nu de waarde 94]
\end{codelines}
\end{NoBreak}

De toelichtingen moeten als informeel commentaar worden beschouwd, waarmee we aan proberen te geven hoe het programma zich gedraagt. Vaak zijn het uitspraken over de toestand waarin het programma zich bevindt, direct na de linker regel te hebben “uitgevoerd”.

\subsection{Basis}

\subsubsection{Declaratie van variabelen}

Een variabele moet altijd eerst worden gedeclareerd. Daarna kan er een waarde aan worden toegekend of kan het op andere manieren worden gebruikt. Een programma waarin variabelen worden gebruikt die nooit zijn gedefinieerd is niet valide. In code fragment~\ref{exa:declaration} staat een voorbeeld van declaratie.

\begin{NoBreak}
\newCodeFragment[exa:declaration][Declaratie van variabelen]
\codeFragmentCaption
\begin{codelines}
  \codeLine{}[\id{x} bestaat (nog) niet]
  \codeLine{\LOCAL \id{x}}[\id{x} heeft nog geen waarde, maar is wel gedefiniëerd]
  \codeLine{\id{x} = 5}[\id{x} bevat nu de waarde 5]
\end{codelines}
\end{NoBreak}

Het concept van declaratie is juist in deze taal heel belangrijk, gezien het lexicaal bereik van variabelen. Wat lexicaal bereik precies inhoudt wordt weldra behandeld.

\subsubsection{Types}
\label{subsec:taal-types}

Variabelen kunnen na declaratie waarden aannemen. Onze taal zal sprkene over drie typen data:
\begin{itemize}
  \item getallen
  \item functies
  \item objecten
\end{itemize}
Een speciale waarde $\nil$ wordt gebruikt om aan te geven dat een variabele nog geen data toegekend heeft gekregen. Het onderscheid tussen deze typen wordt op \emph{dynamisch} niveau gemaakt in plaats van op syntactisch niveau. Dat houdt in dat een willekeurige variabele data van willekeurige typen kan aannemen. Ook kan het in zijn levensduur data van meerdere typen bevatten. Code fragment~\ref{exa:types} geeft dit weer.

\begin{NoBreak}
\newCodeFragment[exa:types][Data van verschillende typen]
\codeFragmentCaption
\begin{codelines}
  \codeLine{\LOCAL \id{x}}[declaratie zonder type indicatie]
  \codeLine{\id{x} = 5}[type “getal”]
  \codeLine{\id{x} = \FUN(\;)\{\dots\}}[type “functie”]
  \codeLine{\id{x} \OBJECT}[type “object”, aan \id{x} kunnen nu attributen worden toegevoegd]
\end{codelines}
\end{NoBreak}

\subsection{Lexicaal bereik}

Het \emph{bereik} (ook wel \emph{scope}) van een variabele, is dat deel van het programma waarin zij zichtbaar is. Er zijn verschillende manieren om dit bereik te definiëren. Een daarvan is \emph{lexicaal bereik} (ook wel \emph{lexical} of \emph{static scoping}) dat expliciet gebruikt wordt door JavaScript \citep{javascript}.
%Gegeven een bepaalde definitie van dit bereik, is de vraag: Als ik een variabele naam tegenkom, over welke variabele heb ik het dan? Code fragmenten \ref{exa:zoek} en \ref{exa:zoek2} illustreren deze vraag. Met \emph{lexicaal bereik} wordt deze dus ruwweg in de “lexicale omgeving” van de referentie gezocht.% T: Dit was idd een beetje rommelig en heb ik herschreven.
De vraag die we ons stellen is: “Als ik de naam van een variabele tegen kom, over welke variabele heb ik het dan?” Code fragmenten \ref{exa:zoek} en \ref{exa:zoek2} illustreren deze vraag. We zien dat het sleutelwoord $\LOCAL$ hier een cruciale rol in speelt. De plaats waar een variabele gedeclareerd is, geeft zijn bereik aan. Wanneer een variabele niet in het lokale bereik is gedeclareerd, zoeken we die op in het \emph{omliggende bereik}: het bereik dat lexicaal gezien om het locale bereik heen ligt.
Door het nesten van bereiken ontstaat een \emph{boomstructuur}. De niveaus van inspringing in onderstaande voorbeelden komt overeen met de boomstructuur van de bereiken.% T: ook dit kan mooier, maar de eerdere tekst is warrig en past niet in de stijl

\begin{NoBreak}
\newCodeFragment[exa:zoek][Zoek de definitie van variabele \id{x}]
\codeFragmentCaption
\begin{codelines}
  \codeLine{\LOCAL \id{x}}[dit is de gezochte definitie van \id{x}]
  \codeLine{\id{x} = 42}
  \codeLine{}
  \codeLine{\LOCAL \id{f}}
  \codeLine{\id{f} = \FUN(\;)}
  \codeLine{\IN \dots \id{x} \dots}[waar is deze \id{x} gedefiniëerd?]
\end{codelines}
\end{NoBreak}

\begin{NoBreak}
\newCodeFragment[exa:zoek2][Zoek de definitie van variabele \id{x}]
\codeFragmentCaption
\begin{codelines}
  \codeLine{\LOCAL \id{x}}
  \codeLine{\id{x} = 42}
  \codeLine{}
  \codeLine{\LOCAL \id{f}}
  \codeLine{\id{f} = \FUN(\;)}
  \codeLine{\IN \LOCAL \id{x}}[dit is de gezochte definitie van \id{x}]
  \codeLine{\IN \id{x} = 43}
  \codeLine{\IN \dots \id{x} \dots}[waar is deze \id{x} gedefiniëerd?]
\end{codelines}
\end{NoBreak}

%Als je de code zó indenteert zoals in bovenstaande twee code fragmenten (\ref{exa:zoek} en \ref{exa:zoek2}): bij elke functie definitie wordt een regel geïndenteerd, dan komt de indentatie ongeveer overeen met de boomstructuur van de scopes. Het moge duidelijk zijn dat als men “naar buiten toe” zoekt naar de definities van variabelen, een zekere boomstructuur van toepassing is.

Een nieuw bereik ontstaat in onze taal enkel bij functie applicatie. Een functie op zich is een \emph{primitieve waarde}, wat betekent dat er “niks gebeurt” als een functie wordt gedefiniëerd, net als er niks gebeurt wanneer je een getal aan een variabele toekent. Bij functie applicatie, echter, wordt een nieuw bereik aangemaakt, met als omliggend bereik het bereik waarin de functie was gedefiniëerd. Daarin wordt vervolgens de \emph{body} van de functie uitgevoerd. In code fragment~\ref{exa:counting} wordt het belang van dit proces weergegeven: als nieuwe bereiken worden aangemaakt bij de definitie van functies, zou de uitvoer van $\id{d}(\;)$ 8 zijn in plaats van 43.

\begin{NoBreak}
\newCodeFragment[exa:counting][Het belang van creatie van bereiken bij functie applicatie]
\codeFragmentCaption
\begin{codelines}
  \codeLine{\LOCAL \id{f}}
  \codeLine{\id{f} = \FUN(\id{n}) \RETURNS \id{g}}
  \codeLine{\IN \LOCAL \id{g}}
  \codeLine{\IN \id{g} = \FUN(\;) \RETURNS \id{n}}
  \codeLine{\IN \IN \id{n} = \id{n} + 1}
  \codeLine{}
  \codeLine{\LOCAL \id{c}}
  \codeLine{\id{c} = \id{f}(5)}
  \codeLine{\id{c}()}[de eerste aanroep $\id{c}(\;)$ levert eerst 6 op\dots]
  \codeLine{\id{c}()}[de tweede aanroep $\id{c}(\;)$ levert daarna 7 op\dots]
  \codeLine{}[]
  \codeLine{\id{d} = \id{f}(42)}[$\id{d}(\;)$: 43, 44, 45, 46, \dots]
\end{codelines}
\end{NoBreak}

\subsection{Prototype overerving en object oriëntatie}
\label{sec:taal-prototypen}

Prototype overerving is een variant van object-geörienteerd programmeren. De kern van object-geörienteerd programmeren is het concept van een \emph{object}, dat ertoe dient een verschijnsel uit de werkelijkheid na te bootsen (een reëel object, een patroon, een abstract idee). Het doel is om meer te kunnen programmeren op een conceptueel niveau. Daarmee wordt bijvoorbeeld zowel creatie als onderhoud van de code makkelijker.

Veel objecten zullen natuurlijk gelijke eigenschappen vertonen, of dezelfde structuur hebben. Verder wilt men concepten als specificering en generalisering toepassen op objecten. Deze problemen kunnen op meerdere manieren worden aangepakt. De bekendste variant is \emph{klasse gebaseerde} object-oriëntatie (ook wel \emph{klassieke object-oriëntatie}) en richt zich op het concept van een \emph{klasse}. Objecten van een bepaalde klasse vertonen de structuur en gedrag van die klasse en heten \emph{instanties}. Van specificering is sprake als een klasse eigenschappen van een andere klasse \emph{overerft}. Klassieke object-oriëntatie vind men in talen als Java en C\# \citep{java,csharp}.

Een andere aanpak met hetzelfde doel is \emph{prototype gebaseerde} object-oriëntatie. Daarbij wordt geen scheiding gemaakt tussen de concepten klasse, die structuur en gedrag specificeert, en instantie, die enkel deze eigenschappen vertoont. In plaats daarvan wordt gewerkt met een prototype structuur, waarbij elk object naar een bepaald \emph{prototype}-object refereert. Nu zijn objecten zelf de dragers van structuur en gedrag.
%Deze methode kan als flexibeler worden gezien, maar ook als een wat minder reëel beeld van de werkelijkheid worden opgevat.

Technisch gezien werkt prototype overerving als volgt. Van elk object is een prototype bekend, of het heeft geen prototype. Wanneer men een attribuut opvraagt van een zeker object, kan de op te leveren waarde procedureel als volgt worden opgevat:

\begin{enumerate}
  \item Bekijk of het attribuut gedefinieerd is in het object zelf. In dat geval weten we de waarde en leveren deze op.
  \item Anders zoeken we het attribuut op in het prototype van het object. Ook dan weten we de waarde en leveren deze op.
  \item Wanneer ook het prototype het attribuut niet bevat, herhalen we de zoektocht voor alle volgende prototypen totdat we het attribuut hebben gevonden.
\end{enumerate}

Het grote verschil tussen object-gebaseerde talen en prototype-gebaseerde talen is dus dat de tweede geen onderscheid maakt tussen klassen en instanties. Een prototype heeft beide functies. Neem bijvoorbeeld het object \id{Deur}:

\newCodeFragment

\begin{codelines}
  \codeLine{\LOCAL \id{Deur}}
  \codeLine{\id{Deur} \OBJECT}
\end{codelines}

We declareren eerst een locale variabele die we vervolgens initialiseren als een object. Vanaf nu kunnen we \id{Deur} als instantie gebruiken door een attribuut te zetten:

\begin{codelines}
  \codeLine{\id{Deur}.\id{open} = 1}
\end{codelines}

Een \id{Deur} is standaard open. We kunnen \id{Deur} ook als een prototype gebruiken. In prototype-gebaseerde talen heet dit \emph{klonen}:

\begin{codelines}
  \codeLine{\LOCAL \id{GeslotenDeur}}
  \codeLine{\id{GeslotenDeur} \OBJECT}
	\codeLine{\id{GeslotenDeur} \CLONES \id{Deur}}
\end{codelines}

\id{GeslotenDeur} heeft dan alle attributen van \id{Deur}. Zo heeft $\id{GeslotenDeur}.\id{open}$ netjes waarde $1$.
Maar een \id{GeslotenDeur} moet natuurlijk gesloten zijn. We zetten zijn attribuut \id{open} op $0$:

\begin{codelines}
	\codeLine{\id{GeslotenDeur}.\id{open} = 0}
\end{codelines}

Een gewone \id{Deur} is nog steeds open. Met andere woorden $\id{Deur}.\id{open}$ heeft nog steeds waarde $1$.
Attributen worden dus per object bewaard. Door \id{open} op $0$ te zetten in \id{GeslotenDeur} verandert er niks in \id{Deur}.

We kunnen net zoveel klonen maken van een object als we willen en net zo diep klonen als we willen. Neem een \id{GlazenDeur}, dit is natuurlijk ook een \id{Deur}, maar wel doorzichtig:

\begin{codelines}
  \codeLine{\LOCAL \id{GlazenDeur}}
  \codeLine{\id{GlazenDeur} \OBJECT}
  \codeLine{\id{GlazenDeur} \CLONES \id{Deur}}
  \codeLine{\id{GlazenDeur}.\id{doorzichtig} = 1}
\end{codelines}

Een gewone \id{Deur} heeft het attribuut \id{doorzichtig} niet, en dus een \id{GeslotenDeur} ook niet. Wanneer we bijvoorbeeld \id{GeslotenDeur}.\id{doorzichtig} levert dit geen waarde op, dit is niet gedefinieerd.
Maar we kunnen besluiten dat deuren standaard niet doorzichtig zijn:

\begin{codelines}
  \codeLine{\id{Deur}.\id{doorzichtig} = 0}
\end{codelines}

\id{GeslotenDeur}en zijn nu ook niet doorzichtig. $\id{GeslotenDeur}.\id{doorzichtig}$ \emph{erft} de eigenschap van \id{Deur} en heeft dus waarde $0$. Maar er geldt nog steeds dat $\id{GlazenDeur}.\id{doorzichtig}$ waarde $1$ heeft.

We zien dat we met prototypes een zeer flexibele methode hebben om object-geörienteerd te programmeren. Het is niet nodig om de compiler of parser van te voren uit te leggen dat objecten aan bepaalde “blauwdrukken” moeten voldoen. We creëren objecten “on-the-fly”, alsmede hun attributen en relaties. Deze methode komt terug in talen als JavaScript, IO en Self \citep{javascript,io,self}.

Natuurlijk is het ook mogelijk om \emph{methoden} te definiëren. Dit zijn functie attributen gekoppeld aan een specifiek object. Stel dat we een \id{GeslotenDeur} graag open willen maken. We definiëren:

\begin{codelines}
  \codeLine{\id{GeslotenDeur}.\id{ontsluit} = \FUN (\id{poging})}
  \codeLine{\IN \IF (\id{poging} = \THIS.\id{code}) \THEN}
  \codeLine{\IN \IN \THIS.\id{open} = 1}
  \codeLine{\IN \ELSE}
  \codeLine{\IN \IN \THIS.\id{open} = 0}
\end{codelines}

\id{this} is hier een expliciete verwijzing naar het huidige object. Op dit moment kunnen we \id{ontsluit} nog niet aanroepen op \id{GeslotenDeur}.
Het attribuut \id{code} is immers niet gedefinieerd in \id{GeslotenDeur} noch in zijn prototype \id{Deur}.

We kunnen natuurlijk een \id{code} toekennen aan \id{GeslotenDeur}, maar laten we een specifieke \id{GeslotenDeur} maken met een \id{code}:

\begin{codelines}
  \codeLine{\LOCAL \id{Kluis}}
  \codeLine{\id{Kluis} \OBJECT}
  \codeLine{\id{Kluis} \CLONES \id{GeslotenDeur}}
  \codeLine{\id{Kluis}.\id{code} = 4321}
\end{codelines}

Wanneer we de methode \id{ontsluit} aanroepen is deze niet gedefinieerd in \id{Kluis}, maar wel in zijn prototype \id{GeslotenDeur}. Die wordt dan uitgevoerd. Een belangrijke observatie is dat \id{ontsluit} wel wordt aangeroepen op \id{Kluis}. Dat betekent dat \id{this} verwijst naar \id{Kluis} en niet \id{GeslotenDeur}. Het attribuut \id{code} wordt dan wel gevonden:

\begin{codelines}
  \codeLine{\id{Kluis}.\id{ontsluit}(1234)}
\end{codelines}

Wanneer we nu $\id{Kluis}.\id{open}$ evalueren komt hier $0$ uit, we hebben de verkeerde code ingevoerd! We proberen het nog een keer:

\begin{codelines}
  \codeLine{\id{Kluis}.\id{ontsluit}(4321)}
\end{codelines}

Deze code is wel correct en $\id{Kluis}.\id{open}$ heeft nu waarde $1$.

\section{Formele definitie}
\label{sec:formeletaal}

%TODO: N of Z? Wel of geen min?
Nu volgt een formele definitie van de syntaxis van de taal, aan de hand van een \BNF\ grammatica. Getallen zijn als volgt gedefiniëerd:
%
\begin{align*}
  \Num \GrammarDef (0 \mid 1 \mid 2 \mid 3 \mid 4 \mid 5 \mid 6 \mid 7 \mid 8 \mid 9)^+
\end{align*}
%
Eigenlijk gebruiken we geen strikte \BNF, in deze specifieke gevallen, maar een hele simpele variant, zoals \textsc{e-bnf}, die ook reguliere expressies toelaat. Bovenstaand voorbeeld maakt dit duidelijk. Voorbeelden van elementen uit $\Num$ zijn “0”, “1”, “235783” en “0003”. Voorbeelden van elementen die niet in $\Num$ zitten zijn “”, “-6”, “4.2”.

\emph{Identifiers}, die gebruikt worden als namen voor variabelen en attributen, zijn op eenzelfde manier als volgt gedefiniëerd:
%
\begin{align*}
  \Id \GrammarDef (a \mid b \mid c \mid \dots \mid A \mid B \mid C \mid \dots)^+
\end{align*}
%
Hierbij moet men zich voorstellen dat alle letters uit het alfabet in de grammaticaregel staan op de voor de hand liggende manier.

Het is soms ook nodig om meerdere komma-gescheiden namen te gebruiken, of een mogelijk lege lijst, zoals bij functie definities. Vandaar de volgende twee productieregels:
%
\begin{align*}
  \Ids \GrammarDef \Id \mid \Ids, \Id \\
  \MaybeIds \GrammarDef \varepsilon \mid \Ids
\end{align*}
%
Een \emph{pad} is een opeenvolging van \Ids\ gescheiden door punten en wordt gebruikt om ook naar attributen van objecten te kunnen refereren:
%
\begin{align*}
  \Path \GrammarDef \Id \mid \Id . \Path
\end{align*}
%
We introduceren geen speciale syntaxis om naar het \emph{this} object te refereren, maar zullen een pad dat begint met de identifier “this” beschouwen als een pad binnen het \emph{this} object. Merk op dat als een “this” op een andere plek in een pad wordt gebruikt, er wel gewoon gesproken wordt over zeker “this” attribuut van een zeker object: alleen vooraan een pad is deze speciale betekenis van toepassing.

\emph{Expressies}, die ofwel primitieve waarden (getallen en functies), ofwel objecten kunnen weergeven, en \emph{boolse expressies}, die gebruikt worden voor loops en conditionele executie, definiëren we als volgt:
%
\begin{align*}
  \Expr \GrammarDef \Num \mid \Expr\; (+ \mid - \mid \times \,)\; \Expr \mid \Path \\
  \GrammarOr \FUN\pmb{(}\,\MaybeIds\,\pmb{)}\; \GrammarOpt{\RETURNS \Id\;} \pmb{\{}\; \Stm\; \pmb{\}} \\
  \Exprs \GrammarDef \Expr \mid \Exprs, \Expr \\
  \MaybeExprs \GrammarDef \varepsilon \mid \Exprs \\
  \BExpr \GrammarDef \TRUE \mid \FALSE \\
  \GrammarOr \BExpr\; (\AND \mid \OR)\; \BExpr \\
  \GrammarOr \NOT\; \BExpr \\
  \GrammarOr \Expr\; (=\: \mid\: \neq\: \mid\: <\: \mid\: \le\: \mid\: >\: \mid\: \ge)\; \Expr
\end{align*}
%
De productieregel voor \emph{statements} is de kern van de grammatica. Een statement is een programma van goede vorm. Het betekent niet noodzakelijk dat het programma \emph{valide} is, maar alle valide programma's zitten wel in de syntactische verzameling $\Stm$. (Vanwege de focus van dit werkstuk definiëren we niet precies wanneer een programma valide is en wanneer niet.)
%
\begin{align*}
  \Stm \GrammarDef \SKIP \\
  \GrammarOr \Stm; \Stm \\
  \GrammarOr \IF \BExpr \THEN \Stm \ELSE \Stm \\
  \GrammarOr \WHILE \BExpr \DO \Stm \\
  \GrammarOr \LOCAL \Id \\
  \GrammarOr \Id \OBJECT \\
  \GrammarOr \Id \CLONES \Id \\
  \GrammarOr \Path = \Expr \\
  \GrammarOr \GrammarOpt{\Path =} \Id\,\pmb{(}\,\MaybeExprs\,\pmb{)}
\end{align*}
%
Merk op dat in delen van deze productieregel \Id's staan waar men misschien een \Path\ had verwacht. Zo zou het wenselijk lijken om bijvoorbeeld “$\id{a}.\id{b} \CLONES \id{c}.\id{d}$” als een programma van goede vorm te beschouwen. Er zijn twee redenen waarom we dit niet hebben gedaan. Allereerst worden de axioma's en deductieregels ingewikkelder en daarmee minder elegant, of er zijn er meer nodig. Maar belangrijker nog: het is niet essentieel voor de taal. Voor elk programma zoals “$\id{a}.\id{b} \CLONES \id{c}.\id{d}$”, bestaat er een equivalent programma zónder zulke paden:

\begin{htmlskip}
  \begin{minipage}[t]{.5\textwidth}
    \newCodeFragment
    \codeLines{
      \codeLine{\id{a}.\id{b} \CLONES \id{c}.\id{d}}
    }
  \end{minipage}
  \begin{minipage}[t]{.5\textwidth}
    \newCodeFragment
    \codeLines{
      \codeLine{\LOCAL x}
      \codeLine{x = \id{a}.\id{b}}
      \codeLine{}
      \codeLine{\LOCAL y}
      \codeLine{y = \id{c}.\id{d}}
      \codeLine{}
      \codeLine{x \CLONES y}
    }
  \end{minipage}
\end{htmlskip}

Hierin moeten $x$ en $y$ “vers” gekozen worden.

% vim: syn=latex spell spl=nl cole=1
