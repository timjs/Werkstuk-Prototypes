\documentclass[11pt,oneside,parskip=half]{scrbook}
\usepackage[a4paper]{geometry}

\usepackage[dutch]{babel}
\usepackage[T1]{fontenc}
\usepackage[utf8]{inputenc}

\usepackage{amsmath}
\usepackage{amssymb}

\usepackage{lmodern}
%\usepackage{millennial}
%\usepackage{tgadventor}

\usepackage{tabularx}% tables: equal width columns
\usepackage{multirow}% tables: rowspan
\usepackage{booktabs}% tables: \toprule, \bottomrule, ...
\usepackage{graphicx}
\usepackage{xcolor}
\usepackage{xargs}
\usepackage{enumitem}
\usepackage{float}
\usepackage{ifthen}
\usepackage[font=small,it]{caption}
\usepackage{environ}
\usepackage{setspace}
\usepackage{multicol}
%\doublespacing

\usepackage{bussproofs}
\def\extraVskip{4pt} % Iets meer verticale ruimte
% De volgende commands gebruik ik om op een simpele manier (1) de toegepaste regel aan de rechterkant van de deductieregel te laten zien, en (2) een automatische nummering aan de linkerkant te laten zien.
% Normaal gesproken gebruik je \RightLabel en \LeftLabel, maar nu gebruik ik dus: \Label (enkel een label) en \CountedLabel (label en automatische nummering)
% Terugzetten op 1 kan met "\setcounter{theLabel}{1}", bijvoorbeeld bij een nieuwe boom
\newcounter{theLabel}
\setcounter{theLabel}{1}
\newcommand{\Label}[1]{
  \RightLabel{\small #1}
}
\newcommand{\CountedLabel}[1]{
  \RightLabel{\small #1}
  \LeftLabel{\scriptsize \arabic{theLabel}}
  \addtocounter{theLabel}{1}
}
% Op een of andere manier moet je in je deductiebomen, de states gaan aangeven. Ik doe dit met doosjes, waarin net genoeg staat dat je iets intuitiever kunt zien wat welke state is, en wat er allemaal in zit.
\newcommand{\stateBox}[1]{%
  \boxed{\mathit{\vphantom{f}#1}}
}
\newcommand{\emptyStateBox}{\stateBox{\phantom{f}}}

\usepackage{tikz}
\usetikzlibrary{calc}
\tikzstyle{every picture}=[thick]

\usepackage{listings}
\lstdefinelanguage{proto}
  {keywords={local, object, clones, function, returns, is,
             print, skip,
             if, then, else, while, do},
   emph={true, false, this, proto},
   %literate={==}{{$=$}}1 {/=}{{$\neq$}}1 {<=}{{$\leq$}}1 {>=}{{$\geq$}}1
   %         {+}{{$+$}}1 {-}{{$-$}}1 {*}{{$\times$}}1 {/}{{$/$}}1
   %         {&}{{$\wedge$}}1 {|}{{$\vee$}}1 {~}{{$\lnot$}}1,
   comment=[l]{--}}
\lstMakeShortInline @ %@
\lstset
  {language=proto,
   basicstyle=\small\ttfamily,
   keywordstyle=\bfseries,
   emphstyle=\bfseries,
   numbers=left,
   numberstyle=\scriptsize\color{gray},
   numbersep=0.5em,
   xleftmargin=1.5em,
   mathescape=true,
   extendedchars=true}
\renewcommand{\lstlistingname}{Voorbeeld}

\usepackage[nounderscore]{syntax}
\setlength{\grammarindent}{8em}
\renewcommand{\syntleft}{\normalfont\itshape}
\renewcommand{\syntright}{}

\newcommand{\newmathcommand}[2]
  {\newcommand{#1}
    {\ensuremath{#2}}}
\newcommand{\newkeyword}[2]
  {\newcommand{#1}
    {\text{\small\ttfamily\bfseries #2}}}
\newcommand{\newconstant}[2]
  {\newkeyword{#1}{#2}}
  %{\newcommand{#1}
     %{\text{\small\ttfamily\scshape #2}}}
\newcommand{\newnonterminal}[2]
  {\newcommand{#1}
    {\textit{#2}}}
\newcommand{\newset}[1]
  {\expandafter\newcommand\expandafter{\csname #1#1\endcsname}
    {\ensuremath{\mathbb{#1}}}}
\newcommand{\renewset}[1]
  {\expandafter\renewcommand\expandafter{\csname #1#1\endcsname}
    {\ensuremath{\mathbb{#1}}}}
\newcommand{\newbool}[1]
  {\expandafter\newcommand\expandafter{\csname #1\endcsname}
    {\ensuremath{\mathbf{#1}}}}

\newcommand{\id}[1]
  {\text{\ttfamily\small\upshape #1}}

\newcommand{\COMP}{;\;}
\newkeyword{\LOCAL}{local\;}
\newkeyword{\VAR}{\LOCAL}
\newkeyword{\OBJ}{obj\;}
\newkeyword{\CLONES}{\;clones\;}
\newkeyword{\OBJECT}{\;object}
\newkeyword{\FUN}{function\;}
\newkeyword{\RETURNS}{\;returns\;}
\newkeyword{\IS}{\;is\;}
\newkeyword{\PRINT}{print\;}
\newkeyword{\SKIP}{skip}
\newkeyword{\IF}{if\;}
\newkeyword{\THEN}{\;then\;}
\newkeyword{\ELSE}{\;else\;}
\newkeyword{\WHILE}{while\;}
\newkeyword{\DO}{\;do\;}
\newkeyword{\END}{\;end}

\newconstant{\TRUE}{true}
\newconstant{\FALSE}{false}
\newconstant{\SELF}{self}
\newconstant{\THIS}{this}
\newconstant{\PROTO}{proto}

\newconstant{\AND}{and}
\newconstant{\OR}{or}
\newconstant{\NOT}{not}

\newnonterminal{\Stm}{Statement}
\newnonterminal{\Id}{Identifier}
\newnonterminal{\Ids}{Identifiers}
\newnonterminal{\MaybeIds}{MaybeIdentifiers}
\newnonterminal{\Path}{Path}
\newnonterminal{\Expr}{Expression}
\newnonterminal{\Exprs}{Expressions}
\newnonterminal{\MaybeExprs}{MaybeExpressions}
\newnonterminal{\BExpr}{BooleanExpression}
\newnonterminal{\Num}{Number}

\renewset{S}
\newset{O}
\newset{B}
\newset{M}
\newset{L}
\newset{F}
\newset{V}
\newset{N}

\newbool{T}
\newbool{F}

\newmathcommand{\LLs}
  {\LL_{\text{s}}}
\newmathcommand{\LLo}
  {\LL_{\text{o}}}
\newmathcommand{\MMs}
  {\MM_{\text{s}}}
\newmathcommand{\MMo}
  {\MM_{\text{o}}}

\newmathcommand{\ms}
  {m_{\text{s}}}
\newmathcommand{\mo}
  {m_{\text{o}}}

\newcommand{\BNF}
  {\textsc{bnf}}

\def\GrammarDef{::=\;&}
\def\GrammarOr{\mid\;&}
\newcommand{\GrammarOpt}[1]{[#1]\;}

\def\IN{\quad}

\newcommand{\<}
  {\ensuremath{\langle}}
\renewcommand{\>}
  {\ensuremath{\rangle}}

\frenchspacing% Cruciaal!
\raggedbottom% Handig

\floatstyle{plain}
\newfloat{code}{H}{code.list.aux}[chapter]
\floatname{code}{Code fragment}

\newcounter{CodeFragment}[chapter]
\renewcommand{\theCodeFragment}{\thechapter.\arabic{CodeFragment}}
\NewEnviron{CodeFragment}[1][]{%
  \refstepcounter{CodeFragment}%
  %\setcounter{CodeFragmentLineNo}{0}%
  \vspace{-1pc}%
  \begin{code}
    \ifthenelse{\equal{#1}{}}{\caption{}}{\caption{#1}}%
	  \begin{equation*}%
		  \begin{array}{l@{\hspace*{.02\textwidth}}|@{\hspace*{.02\textwidth}}l}%
		  \hspace*{.35\textwidth} & \hspace*{.55\textwidth} \\[-1pc]%
		  \BODY%
		  \end{array}%
	  \end{equation*}%
	  \vspace{-1pc}%
	\end{code}%
  \vspace{-1pc}%
}

\newcounter{CodeFragmentLineNo}[CodeFragment]
\newcommand{\Line}[2]{%
  \refstepcounter{CodeFragmentLineNo}%
  \text{\color{gray} \scriptsize \arabic{CodeFragmentLineNo}}\;\;%
  #1 &%
  \ifthenelse{\equal{#2}{}}{}{\textit{\small- #2}}%
  \\
}

% todo if second param empty? then don't display scalebox etc..
\newcommand{\AxiomOrInference}[2]{%
  \renewcommand{\arraystretch}{1.2}%
  \begin{minipage}{\textwidth}% prevent page break
    #1% = \begin{prooftree} ... \end{prooftree}
    \vspace*{-.5pc}%
    \hspace*{.45\textwidth}% ideally: .5\textwidth - 1/2 "desda\quad"
    \scalebox{.85}{%
      #2%
    }%
  \end{minipage}%
}

\NewEnviron{NSAxiom}[1]{%
  \renewcommand{\arraystretch}{1.2}%
  \begin{minipage}{\textwidth}% prevent page break
  \BODY%
  \end{minipage}%
}
\NewEnviron{NSConditions}{%
  \vspace*{-.5pc}%
  \hspace*{.45\textwidth}% ideally: .5\textwidth - 1/2 "desda\quad"
  \scalebox{.85}{%
    \begin{tabular}{@{}r@{\quad}l@{}}%@
      \textbf{desda} \BODY%
    \end{tabular}%
  }%
}
\newcommand{\Cond}[1]{%
  & #1 \\
}

\newcommand{\FiniteFunctions}[2]{\ensuremath{\text{``eindige''~}{#1}^{#2}}}

\newcommand{\surrAngle}[1]{\ensuremath{\big<\,#1\,\big>}}
\newcommand{\SurrAngle}[1]{\ensuremath{\big<\!\big<\,#1\,\big>\!\big>}}
%\newcommand{\SurrAngle}[1]{\ensuremath{\big<\!\big|\,#1\,|\!\big>}}
\newcommand{\surr}[1]{\ensuremath{\big[\,#1\,\big]}}
\newcommand{\Surr}[1]{\ensuremath{\big[\!\big[\,#1\,\big]\!\big]}}

% configuration: " <S, m, \sigma, \tau> "
%\newcommand{\Config}[2]{\left<#1, #2\right>}%  -- normaal
%\newcommand{\Config}[2]{\surr{#1}_{#2}}%       -- [ ]m,s,t
%\newcommand{\Config}[2]{\surrAngle{#1}_{#2}}%  -- < >m,s,t
\newcommand{\Config}[2]{\SurrAngle{#1}_{#2}}%  -- << >>m,s,t % T: als test... weet niet hoe het er met millenial uit ziet

\def\BOT{\rotatebox{90}{$\Vdash$}}
\let\nil\BOT

\def\DEF{\buildrel{\text{def}}\over{=}}
\def\Proto{\sqsubset^p}
\def\Outer{\sqsubset^s}
\def\ProtoEq{\sqsubseteq^p}
\def\OuterEq{\sqsubseteq^s}
\def\Attr{\textit{attr}}
\def\SDef{\textit{def}}
\newcommand{\ScopeID}[1]{\ensuremath{(1, #1)}}
\newcommand{\ObjectID}[1]{\ensuremath{(2, #1)}}

\title{Een natuurlijke semantiek voor prototype oververing en lexicaal bereik}
\author{Kelley van Evert \& Tim Steenvoorden}

% Code fragments
% Usage:
%   \newCodeFragment[fraglabel][caption]
%   \codeFragmentCaption
%   \codeLines{
%     \codeLine           {\VAR f}
%     \codeLine[linelabel]{\VAR f}
%     \codeLine           {\VAR f}[comment]
%     \codeLine[linelabel]{\VAR f}[comment]
%   }
%   \codeLines{
%     \codeLine{\VAR f}
%   }
%   \ref{fraglabel} -- e.g. "3.1" for the first code fragment in chapter 3
%   \ref{linelabel} -- e.g. "6" for the sixth line in the code fragment
\newcounter{ncode}[chapter]
\renewcommand{\thencode}{\thechapter.\arabic{ncode}}
\newcommand{\theCodeFragmentCaption}{}
\newcounter{codeline}[ncode]
\newcommandx{\newCodeFragment}[2][1=,2=]{% args: [1] label; [2] title = caption
  \refstepcounter{ncode}%
  \ifthenelse{\equal{#1}{}}{}{\label{#1}}%
  \renewcommand{\theCodeFragmentCaption}{#2}%
}
\newcommand{\codeFragmentCaption}{%
  ~\centerline{\small%
    \textit{Code fragment \thencode}%
    \ifthenelse{\equal{\theCodeFragmentCaption}{}}%
    {}%
    {. \theCodeFragmentCaption}%
  }~%
  \\[-1.6pc]%
}
\newcommand{\codeLines}[1]{%
  \par
  \vspace{4pt}%
  \begin{tabular}{@{~~}r@{~}!{\color{black!70} \vrule width 1.5pt}@{~~}l@{~~}l@{}}%
    #1%
  \end{tabular}%
  \vspace{4pt}%
  \par
}
\newcommandx{\codeLine}[3][1=,3=]{% args: [1] label; [2] code (math); [3] comment
  \refstepcounter{codeline}%
  \ifthenelse{\equal{#1}{}}{}{\label{#1}}%
  \lineNum{\thecodeline} &%
  $ #2 $ &%
  \ifthenelse{\equal{#3}{}}{}{\lineComment{#3}} \\[-1pt]%
}
\newcommand{\lineNum}[1]{%
  {\color{black!50}\scriptsize #1}%
}
\newcommand{\lineComment}[1]{%
  \textit{--- \small #1}%
}

\begin{document}

\maketitle

\frontmatter

\tableofcontents

\mainmatter

% !TeX root = werkstuk.tex
% !TeX program = xelatex

\chapter{Inleiding}

% Wat
In dit werkstuk presenteren we een natuurlijke semantiek die wij ontworpen hebben om de concepten \emph{lexicaal bereik} en \emph{prototype overerving} in object-geörienteerde talen te karakteriseren. Daartoe hebben we een minimale taal ontworpen die geïnspireerd is op de bestaande programmeertalen JavaScript en IO. JavaScript -- ook wel bekend als \emph{ECMAScript}, de naam van de standaard -- is een dynamische, prototype-gebaseerde taal die veelvuldig wordt gebruikt bij het ontwikkelen van internettoepassingen \citep{javascript}. Een opvallende functie van JavaScript is het algemeen gebruik van lexicaal bereik. (Vele imperatieve talen gebruiken lexicaal bereik, maar dit vaak met beperkingen -- in JavaScript is sprake van ``puur'' lexicaal bereik.) IO is een onderzoekstaal door Steve Dekorte \citep{io}. Het belangrijkste kenmerk van deze taal is het prototype-gebaseerde object model.

% Waarom
Lexicaal bereik (ook wel \emph{static scoping} genaamd) en prototype overerving zijn mooie fenomenen. Ze zijn ook de fundamenten van “The World's Most Misunderstood Programming Language”: JavaScript. Maar lexicaal bereik ligt mensen eigenlijk heel natuurlijk: zo redeneren wiskundigen al meer dan honderd jaar met formules waarin variabelen lexicaal bereik hebben. En prototype overerving is slechts een elegant en simpel alternatief op klassieke overerving, wanneer het gaat om object-geörienteerd programmeren.

% Doel
Het doel van dit werkstuk is daarom een formele betekenis te geven aan deze concepten, maar dan wel zó dat de interpretatie van de formele uitspraken zo natuurlijk mogelijk en conceptueel verantwoord is. De bedoeling is dat men de gewoon Nederlandse interpretatie van een willekeurig axioma of deductieregel tegen zou kunnen komen in een college programmeren:

\begin{htmlskip}
  \begin{multicols}{2}
    \small
    \raggedcolumns
    \setlength{\columnseprule}{.5pt}
    \begin{minipage}{.25\textwidth}\vspace*{-1.1pc}%
\begin{gather*}
  \Terminates{i \OBJECT}
    {\ms, \mo}
    {\sigma, \tau}
    {\ms', \mo'} \\
  \begin{array}{r@{}l}%@
    \textbf{desda } & {\Finds(\ms, \sigma, i) = \sigma'} \\
    &{\omega = |\mo|} \\
    &{\ms' = \ms \surr{ \sigma' \mapsto \big(b_{\ms(\sigma')}[i\mapsto \omega], \pi_{\ms(\sigma')}\big) }} \\
    &{\mo' = \mo : (\varnothing,\nil)}
  \end{array}
\end{gather*}%
    \end{minipage}
    
    \columnbreak
    
    \textit{“Zoals jullie weten, moeten we bij statische bereiken eerst de definitie van de variabele zoeken in de huidige en daarna omliggende bereiken. Daarna maken we ruimte vrij in het geheugen en kan een nieuw object worden gemaakt. Een verwijzing naar dit object wordt vervolgens in de variabele gestopt\dots”}
  \end{multicols}
\end{htmlskip}
\begin{htmlonly}
  <div class="cols cols-2">
    <div class="col" style="width:63%">
      \begin{gather*}
        \Config{i \OBJECT}{\ms, \mo, \sigma, \tau}
        \longrightarrow
        \ms', \mo' \\
        \begin{array}{r@{}l}%@
          \textbf{desda } & \Finds(\ms, \sigma, i) = \sigma_\text{def} \\
          & \omega = \Nexto(\mo) \\
          & \ms' = \ms \surr{ \sigma_\text{def} \mapsto \big(b_{\ms(\sigma')}[i\mapsto \omega], p_{\ms(\sigma')}\big) } \\
          & \mo' = \mo \surr{ \omega \mapsto \big(\varnothing, \nil\big) }
        \end{array}
      \end{gather*}
    </div>
    <div class="col" style="width:37%">
      <blockquote>
        “Zoals jullie weten, moeten we bij statische scope eerst de definitie van de variabele zoeken in de huidige
        en daarna omliggende bereiken. Daarna maken we ruimte vrij in het geheugen en kan een nieuw object worden gemaakt.
        Een verwijzing naar dit object wordt vervolgens in de variabele gestopt…”
      </blockquote>
    </div>
    <div style="clear:both;"></div>
  </div>
\end{htmlonly}

% Opzet
Na het bespreken van een aantal notationele keuzes en terminologie, presenteren we eerst de minimale taal, vervolgens het semantische model en tenslotte de natuurlijke semantiek die de twee voorgaande aan elkaar koppelt. We sluiten af met een afleiding voor een kort programma om het gebruik van de ontwikkelde regels en axioma's te demonstreren.

% vim: syn=latex spell spl=nl cole=1 cocu=nv


\chapter{Notatie en terminologie}

\section{Functies}

In dit werkstuk identificeren we een functie met zijn grafiek, d.w.z.~een functie $f : X \to Y$ is gelijk aan de verzameling paren $(x, y) \in X \times Y$ waarvoor geldt: $f(x) = y$.

\section{Tupels}

We zullen meermaals in ons werkstuk gebruik maken van willekeurig grote, maar altijd eindige, ``lijsten'' van elementen uit een zekere verzameling: \emph{tupels}. Deze tupels worden gerepresenteerd door eindige (partiële) functies $t : \sN \to X$, als X de verzameling element in kwestie is, waaraan nog een paar extra voorwaarden worden gesteld. De verzameling van alle tupels op een zekere verzameling $X$, genoteerd $X_{\langle\rangle}$, is als volgt gedefiniëerd:

$$ X_{\langle\rangle} \DEF \big\{\, t : \sN \to X \mid \exists_{N \in \sN} \big[ \forall_{n < N}[t(n) \neq \bot] \land \forall_{n\ge N}[t(n) = \bot] \big] \,\big\} $$

We schrijven $\langle\rangle$, maar ook wel $\varnothing$, voor de lege tupel (deze is natuurlijk hetzelfde voor elke waarden verzameling $X$).

We schrijven $\langle x_0, x_1, \dots, x_{N-1}\rangle$ voor de tupel $t$ waarvoor geldt: $\forall_{n < N}[t(n) = x_n]$, en: $\forall_{n \ge N}[t(n) = \bot]$.

Als $t$ een tupel is $\in X_{\langle\rangle}$, en $x$ een element van $X$, dan schrijven we $t:x$ voor de tupel $t' = t[N \mapsto x]$, waarbij $N = \min\{n\in\sN \mid t(n) = \bot\}$.

\section{Beschouwing semantisch model}

We definieren in dit werkstuk een natuurlijke semantiek, d.w.z.~een ?-ste orde logica, met axioma's en deductieregels, en een bijbehorende structuur waarin deze zich afspeelt.

Deze structuur, die we ook wel het \emph{semantisch model} zullen noemen, heeft onderstaand opgesomde elementen. Deze worden verderop precies gedefinieerd, onderstaande opsomming geeft slechts een algemeen beeld.

\begin{description}
	\item[$\mathbb{M}$]\hfill\\ De verzameling mogelijke \emph{geheugens}, welke ook wel als \emph{eindtoestanden} worden geinterpreteerd.
	\item[$(\mathit{Stm} \times \mathbb{M} \times \mathbb{L} \times \mathbb{L})$]\hfill\\ De verzameling \emph{toestanden}, ook wel \emph{configuraties}.
	\item[$(\longrightarrow)$]\hfill\\ Een tweeplaatsig predikaat welke als eerste argument een element uit de verzameling van toestanden neemt, en als tweede argument een element uit de verzameling van eindtoestanden $(\mathbb{M}\dots)$. De uitspraak $(S, m, \sigma, \tau) \longrightarrow m'$ moet worden geinterpreteerd worden als:
	\begin{quote} ``Het programma $S$, met geheugen $m$, in scope $\sigma$ en met als $\mathbf{this}$ object $\tau$, resulteert in eindtoestand $m'$, mits $S$ \emph{correct} is''. \end{quote}
\end{description}

\section{Notationele conventies}

Terwille van elegantie houden we een aantal gebruikelijke notationele conventies aan:

\begin{enumerate}
	\item Voor elke twee willekeurige tweestemmige predikaten $\mathsf{S}$ en $\mathsf{T}$ (mogelijk ook $=$), en drie willekeurige elementen $a$, $b$ en $c$, definieren we de afkorting: $$a \operatorname{\mathsf{S}} b \operatorname{\mathsf{T}} c \buildrel{\mathrm{def}}\over{=} a \operatorname{\mathsf{S}} b \land b \operatorname{\mathsf{T}} c$$ in het geval dat deze bewering correct getypeerd is.
	\item Op eenzelfde manier definieren we ook de volgende afkorting: $$ \{a \in A \mid \phi \} \buildrel{\mathrm{def}}\over{=} \{a \mid a \in A \mid \phi\}$$
\end{enumerate}

[...]


\chapter{Taal en syntax}

In dit hoofdstuk zullen we de taal presenteren waarvoor we een natuurlijk taal construeren. De taal maakt gebruikt van prototype overerving en lexicaal bereik. Eerst zullen we een aantal voorbeeldprogramma's beschouwen, om zo informeel het karakter van de te formaliseren taal over te brengen. Daarna geven we een rigoreuze definitie met behulp van een BNF grammatica. De structuur van de productieregels van grammatica worden in latere hoofdstukken gebruikt om axioma's en deductieregels op te stellen. Daarmee heeft de grammatica in zekere zin een dubbele functie.

Elk voorbeeldprogramma en zijn toelichtingen worden als volg gepresenteerd:

	\begin{SyntaxExample}
		\N \VAR f & \textit{- variabelen moeten worden gedeclareerd} \\
		\N f = \FUN(i) \RETURNS n \\
		\N \IN \VAR n \\
		\N \IN n = 2 \times (i + 5) \\
		\N & \textit{- x bestaat niet in deze scope} \\
		\N \VAR x & \textit{- x is ongedefinieerd (maar wel aanwezig)} \\
		\N x = f(42) & \textit{- x = 89}
	\end{SyntaxExample}

De toelichtingen moeten als informeel commentaar worden beschouwd, waarmee we aan proberen te geven hoe het programma zich gedraagt. Vaak zijn het uitspraken over de toestand waarin het programma zich bevindt, direct na de linker regel te hebben ``uitgevoerd''.

\section{Voorbeeldprogramma's}

Een variabele moet gedeclareerd worden, en pas daarna kan er een waarde aan worden toegekend.

	\begin{SyntaxExample}
		\N & \textit{- x bestaat niet (in deze scope)} \\
		\N \VAR x & \textit{- x is ongedefinieerd (maar wel aanwezig)} \\
		\N x = 5 & \textit{- x = 5}
	\end{SyntaxExample}

Het concept van declaratie is juist in deze taal, gezien het lexicaal bereik van variabelen, heel belangrijk. Vergelijk het bovenstaande programma fragment bijvoorbeeld met de volgende situatie.

Variabelen hebben geen vaste type. Er zijn drie typen waarden in de taal: getallen, functies en objecten.

	\begin{SyntaxExample}
		\N \VAR x \\
		\N x = 5 & \textit{- de waarde van } x \textit{ is een getal} \\
		\N x = \FUN()\,\{\;\SKIP\;\} & \textit{- de waarde van x is een functie} \\
		\N x \OBJECT & \textit{- de waarde van x is een object}
	\end{SyntaxExample}

De taal is object georienteerd.

	\begin{SyntaxExample}
		\N \VAR o \\
		\N o \OBJECT \\
		\N & \textit{- o.f is niet gedefinieerd} \\
		\N o.f = \FUN()\,\{\;\SKIP\;\} & \textit{- toekenning waarde aan object attribuut} \\
		\N & \textit{- o.f is wel gedefinieerd} \\
		\N o.n = 5 &
	\end{SyntaxExample}

Van de drie typen, zijn getallen en functies \emph{primitief}, en objecten \emph{niet primitief}. Primitieve waarde worden zelf gekopieerd (\emph{by-value}), maar van niet-primitieve waarden worden \emph{referenties} gekopieerd (\emph{by-reference}).

	\begin{SyntaxExample}
		\N \VAR x;\; x = 6 \\
		\N \VAR y;\; y = x & \textit{- x = 6 en y = 6} \\
		\N y = 7 & \textit{- x = 6 en y = 7} \\
		\N \\
		\N \VAR p;\; p.n = 6 \\
		\N \VAR q;\; q = p & \textit{- p en q verwijzen nu naar hetzelfde object} \\
		\N & \textit{- p.n = 6 en q.n = 6} \\
		\N q.n = 7 & \textit{- p.n = 7 en q.n = 7}
	\end{SyntaxExample}

\subsection{Lexical scope}

Als in een zekere scope een variabele wordt gereferenceerd (nog) niet is gedefinieerd, wordt in omliggende scopes ``gezocht'' naar een definitie van deze variabele.

\begin{SyntaxExample}\label{exa:lexical}
		\N \VAR x; \\
		\N \VAR f;\; f = \FUN(i) \\
		\N \IN x = i + 5 \\
		\N \\
		\N f(5) & \textit{- x = 10}
	\end{SyntaxExample}

..maar wanneer deze wel in de huidige scope bestaat, worden omliggende scopes ``met rust gelaten''.

	\begin{SyntaxExample}
		\N \VAR x \\
		\N \VAR f \\
		\N f = \FUN(i) \\
		\N \IN \VAR x \\
		\N \IN x = i + 5 \\
		\N \\
		\N f(5) & \textit{- x heeft nog geen waarde}
	\end{SyntaxExample}

Telkens wanneer een functie wordt aangeroepen, wordt een \emph{nieuwe scope} aangemaakt voor lokale variabelen. Variabelen van deze nieuwe scope kunnen later nog gereferenceerd worden, doordat bijvoorbeeld de functie een lokale functie teruggeeft.

	\begin{SyntaxExample}
		\N \VAR f \\
		\N f = \FUN(n) \RETURNS g \\
		\N \IN \VAR g \\
		\N \IN g = \FUN() \RETURNS n \\
		\N \IN \IN n = n + 1 \\
		\N \\
		\N \VAR c \\
		\N c = f(5) & \textit{- c() $\rightarrow$ 6, 7, 8, \dots}
	\end{SyntaxExample}

\begin{lstlisting}[caption=Een countervoorbeeld,label=exa:counter]
local f
f = function(n) returns g
    local g
    g = function() returns n
        n = n + 1

local c
c = f(5)
c()                          # $\rightarrow 6, 7, 8, \dots$
\end{lstlisting}

\fbox{maar dan wat beter geschreven, etc...}

\subsection{Prototype overerving}

Prototype overerving is een eenvoudige en dynamische variant van object-geörienteerd programmeren. Net als in klassieke object-gebaseerde talen is er sprake van een object waarin \emph{attributen} zijn gedefinieerd. Elk object heeft ook een expliciete \emph{ouder} (of \enquote{parent}). Wanneer we binnen een object een attribuut willen evalueren, doen we dit in drie stappen:
\begin{enumerate}
  \item Bekijk of het attribuut gedefinieerd is in het object zelf. In dat geval weten we de waarde en leveren deze op.
  \item Anders zoeken we het attribuut op in de ouder van het object. Ook dan weten we de waarde en leveren deze op.
  \item Wanneer ook de ouder het attribuut niet bevat, herhalen we de zoektocht voor alle volgende ouders totdat we het attribuut hebben gevonden.
\end{enumerate}
Ook hier is dus sprake van een boomstructuur.

Het grote verschil tussen object-gebaseerde talen en prototype-gebaseerde talen is dat de tweede geen onderscheid maakt tussen \emph{klassen} en \emph{instanties}. Een prototype heeft beide functies. Neem bijvoorbeeld het prototype @Deur@:
\begin{lstlisting}[name=deuren]
local Deur
Deur object
\end{lstlisting}
We kunnen @Deur@ direct als instantie gebruiken door een attribuut te zetten:
\begin{lstlisting}[name=deuren]
Deur.open = 1
\end{lstlisting}
Een @Deur@ is standaard open. Maar we kunnen @Deur@ ook als een klasse gebruiken door ervan te erven. In prototype-gebaseerde talen heet dit \emph{klonen}:
\begin{lstlisting}[name=deuren]
local GeslotenDeur
GeslotenDeur object
GeslotenDeur clones Deur
\end{lstlisting}
@GeslotenDeur@ heeft dan alle attributen van @Deur@:
\begin{lstlisting}[name=deuren]
GeslotenDeur.open            # => 1
\end{lstlisting}
Maar een @GeslotenDeur@ moet natuurlijk gesloten zijn. We zetten zijn attribuut @open@ op @0@:
\begin{lstlisting}[name=deuren]
GeslotenDeur.open = 0
\end{lstlisting}
Een gewone @Deur@ is nog steeds open:
\begin{lstlisting}[name=deuren]
Deur.open                    # => 1
\end{lstlisting}
Attributen worden dus per object bewaard. Door @open@ op @0@ te zetten in @GeslotenDeur@ verandert er niks in @Deur@.

We kunnen net zoveel klonen maken van een object als we willen en net zo diep klonen als we willen. Neem een @GlazenDeur@, dit is natuurlijk ook een @Deur@, maar wel doorzichtig:
\begin{lstlisting}[name=deuren]
local GlazenDeur
GlazenDeur object
GlazenDeur clones Deur
GlazenDeur.doorzichtig = 1
\end{lstlisting}
Een gewone @Deur@ heeft het attribuut @doorzichtig@ niet, en dus een @GeslotenDeur@ ook niet:
\begin{lstlisting}[name=deuren]
GeslotenDeur.doorzichtig     # => fout!
\end{lstlisting}
Maar we kunnen besluiten dat deuren standaard niet doorzichtig zijn:
\begin{lstlisting}[name=deuren]
Deur.doorzichtig = 0
\end{lstlisting}
Zodat ook onze @GeslotenDeur@ niet doorzichtig is:
\begin{lstlisting}[name=deuren]
GeslotenDeur.doorzichtig     # => 0
\end{lstlisting}
Maar er geld nog steeds:
\begin{lstlisting}[name=deuren]
GlazenDeur.doorzichtig       # => 1
\end{lstlisting}

We zien dat we met prototypes een zeer flexibele methode hebben om object-geörienteerd te programmeren. Het is niet nodig om de compiler of parser van te voren uit te leggen dat objecten aan bepaalde \enquote{blauwdrukken} moeten voldoen. We creëren objecten \enquote{on-the-fly}, alsmede hun attributen en relaties. Deze methode komt terug in talen als JavaScript, IO en Self.

Natuurlijk is het ook mogelijk om \emph{methoden} te definiëren. Dit zijn functies die gekoppeld zijn aan een specifiek object. Stel dat we een @GeslotenDeur@ graag open willen maken. We definiëren:
\begin{lstlisting}[name=deuren]
GeslotenDeur.ontsluit = function (poging)
    if poging == this.code then
        this.open = 1
    else
        this.open = 0
\end{lstlisting}
@this@ is hier een expliciete verwijzing naar het huidige object. Op dit moment kunnen we @ontsluit@ nog niet aanroepen op @GeslotenDeur@:
\begin{lstlisting}[name=deuren]
GeslotenDeur.ontsluit(1234)  # => fout!
\end{lstlisting}
Het attribuut @code@ is immers niet gedefinieerd in @GeslotenDeur@ noch in zijn prototype @Deur@.

We kunnen natuurlijk een @code@ toekennen aan @GeslotenDeur@, maar laten we een specifieke @GeslotenDeur@ maken met een @code@:
\begin{lstlisting}[name=deuren]
local Kluis
Kluis object
Kluis clones GeslotenDeur
Kluis.code = 4321
\end{lstlisting}
Wanneer we de methode @ontsluit@ aanroepen is deze niet gedefinieerd in @Kluis@, maar wel in zijn prototype @GeslotenDeur@. Die wordt dan uitgevoerd. Een belangrijke observatie is dat @ontsluit@ wel wordt aangeroepen op @Kluis@. Dat betekent dat @this@ verwijst naar @Kluis@ en niet @GeslotenDeur@. Het attribuut @code@ wordt dan wel gevonden!
\begin{lstlisting}[name=deuren]
Kluis.ontsluit(1234)
Kluis.open                   # => 0
\end{lstlisting}
Oeps\dots Dat was de verkeerde code, nog een poging:
\begin{lstlisting}[name=deuren]
Kluis.ontsluit(4321)
Kluis.open                   # => 1
\end{lstlisting}

\section{Grammatica}

[...en vervolgens helemaal formeel -- even uitleggen van BNF etc..]

% vim: spell spl=nl

\chapter{Semantisch model}

\section{Bindingen}\label{sec:bindinge}

% [Tim:] Onderstaande twee paragrafen heb ik gewijzigd zodat de termen "binding" en  "binding groep" zo gebruikt worden zoals wij dat afgesproken hebben.
Aan de basis van ons model ligt het concept van een \emph{binding}. Een binding is een toekenning van een \emph{waarde} aan een variabele (een element uit de syntactische verzameling \Id\ komt). Bindingen zijn bijvoorbeeld van belang om de gedefiniëerde variabelen binnen een scope vast te leggen, of de attributen van een bepaald object. Een \emph{groep bindingen} is een eindige functie $b : \Id \to \sV$. De verzameling van alle groepen van bindingen definiëren we dus als
% [Tim:] Hier staat dus niet meer dat gekke pijltje van \FiniteFunctions, want nu vermelden we in [Notatie en Terminologie] dat alle functies eindig zullen zijn.
$$ \sB \DEF \sV^\Id $$
% [Tim:] wat denk jij: "sectie 4.4" of "onderdeel 4.4" of gewoon "4.4" ??
We komen later terug op wat de waarden $\sV$ precies zijn in sectie \ref{sec:waarden}. Voor nu is het voldoende om te weten dat in ieder geval de natuurlijke getallen $\sN$ deel uitmaken van $\sV$.

% [Tim:] Ik geloof niet dat we in [Semantisch Model] uitvoerig in moeten gaan op het updaten van functies, aangezien dat een syntactische aangelegenheid is, die we het best kunnen beschrijving in [Notatie en Terminologie], en hier slechts even noemen.

%Een groep bindingen $b \in \sB$ is in eerste instantie leeg. Dit geven we aan met $\emptyset$. We willen natuurlijk \Id's kunnen koppelen aan waardes. Hiervoor voeren we een notatie in om $b$ te \emph{updaten}. Om bijvoorbeeld de waarde $5$ toe te kennen aan de \Id\ $x$ zoals in voorbeeld \ref{exa:todo} schrijven we
%\begin{equation*}
%  b[x \mapsto 5]
%\end{equation*}
%zodat wanneer we $x$ ``opvragen'' in $b$ we weten dat
%\begin{equation*}
%  b(x) = 5.
%\end{equation*}
%Wanneer we meerdere \Id's willen koppelen aan waardes, bijvoorbeeld $y$ aan $7$ en $z$ aan $9$ kan dat met bovenstaande notatie als volgt
%\begin{equation*}
%  \big(b[y \mapsto 7]\big)[z \mapsto 9].
%\end{equation*}
%Wat we afkorten tot
%\begin{equation*}
%  b[y \mapsto 7, z \mapsto 9].
%\end{equation*}

% [Tim:] slechts de tekst wat gereviseerd (passieve vormen, woordherhaling...)
Bindingen komen veelvuldig terug in ons model. In scopes worden \emph{variabelen} gedeclareerd en aan waarden gekoppeld. Bij objecten zijn het de \emph{attributen} die waarden krijgen toegekend.
% [Tim:] ik zou deze regel schappen
% Bij scopes moeten we ook rekening houden met eventuele bindingen in de scope buiten de huidige. Eenzelfde opzet geld voor objecten. Door prototype overerving moeten we op zoek naar een attribuut in het prototype van het huidige object, wanneer het niet gedefinieerd is in het object zelf.

\section{Scope en omliggende scopes}

% [Tim:] ik wil graag "procedureel als volgt opvatten" oid zeggen, om aan te geven dat zo'n 'procedure' niet het enige perspectief is die men kan nemen.
%Zoals in \ref{sec:voorbeelden} informeel is behandeld, zijn scopes goed te representeren met een boomstructuur. Stel we evalueren een variabele $x$ in scope $s$:
In sectie \ref{sec:voorbeelden} is informeel gebleken dat scopes conceptueel goed te zien zijn als een boomstructuur. De evaluatie van een zekere variabele \I{x} in scope $s$ zou dan procedureel als volgt kunnen worden uitgelegd:

\newCodeFragment
\codeLines{
  \codeLine{\I{x}}[We bevinden ons in een zekere scope $s$.]
}

Dan zoeken we \I{x} eerst op in de bindingen groep $b_s$, behorende bij scope $s$.
$$
  b_s(x).
$$
Zoals ook te zien in \ref{exa:todo} hebben we twee mogelijkheden:

\begin{enumerate}
  \item \I{x} is gedefinieerd in $b_s$ en we gebruiken de gevonden waarde.
  \item \I{x} is niet gedefinieerd in $b_s$ en we moeten \I{x} opzoeken in de omliggende scope.
\end{enumerate}

We moeten dus niet alleen de bindingen van de scope zelf bijhouden, maar ook een verwijzing naar zijn \emph{omgevende scope}. Een scope $s$ is definiëren we dus als een paar $(b,\pi)$, met in $b$ de bindingen en $\pi$ een \emph{verwijzing} naar de omgevende scope (ook wel \emph{parent}, of \emph{outer} scope).

We moeten benadrukken dat $\pi$ een \emph{verwijzing} is, en niet een \emph{kopie} van de bindingen groep van de omgevende scope. Stel dat we het programma in code fragment~\ref{exa:lexical} uitvoeren. Op het moment dat we $f()$ aanroepen in regel~\ref{exa:lexical:eerste} willen we dat $x$ daarna evalueert naar de waarde $2$. Evenzo moet $x$ na regel~\ref{exa:lexical:tweede} evalueren naar de waarde $4$. De scope $s_f$ van functie $f$ heeft een eigen binding $b_f$ die gedurende de executie van het programma leeg is, $x$ is namelijk niet gedeclareerd als een \LOCAL variabele. De omgevende scope $\pi_f$ van functie $f$ verwijst naar scope $s$, zodat de variabele $x$ uiteindelijk wel gevonden wordt.

\newCodeFragment[exa:lexical][Lexicale scope: opslaan en vinden van variabelen]
\codeFragmentCaption
\codeLines{
  \codeLine{\VAR \I{x}}
  \codeLine{\I{x} = 1}
  \codeLine{\VAR \I{f}}
  \codeLine[exa:lexical:def]{\I{f} = \FUN()}[Introductie nieuwe scope]
  \codeLine{\IN \I{x} = 2 \times \I{x}}
  \codeLine{}[Einde nieuwe scope]
  \codeLine[exa:lexical:eerste]{\I{f}()}[$\I{x} = 2$]
  \codeLine[exa:lexical:tweede]{\I{f}()}[$\I{x} = 4$]
}

Stel dat we geen verwijzing in de scope opslaan maar een kopie van de omgevende bindingen. Op het moment dat we $f$ definiëren in regel~\ref{exa:lexical:def} is scope $s_f$ een paar $(b_f,p_f)$ met $b_f,p_f\in\sB$. Net als hierboven zijn de eigen bindingen $b_f$ leeg. De binding $p_f$ bevat een functie onder naam $f$ en de waarde $1$ onder naam $x$. Wanneer we $x$ aanpassen door de aanroep in regel~\ref{exa:lexical:eerste} wordt dit doorgevoerd in de binding $p_f$ maar, omdat dit een kopie is, niet in de binding $b_s$ van de omgevende scope $s$. We moeten dus wel een verwijzing opslaan willen we het gevraagde gedrag krijgen. Daarnaast wordt het met kopieën erg lastig om een boomstructuur te creëren zodat we een variabele nog hogerop kunnen opzoeken.

Een scope $s$ is dus een element uit de verzameling
\begin{equation*}
  \sS \DEF \sB \times (\sL_s \cup \{\nil\}).
\end{equation*}
Hierbij zijn $\sB$ de bindingen zoals besproken in §\ref{sec:bindingen}. $\sL_s$ zijn locaties van scopes. Op het begrip locatie komen wij nog terug in §\ref{sec:locaties}. We moeten er wel rekening mee houden dat er een soort ``ultieme'' omgevende scope is. Het kan dus zijn dat een scope geen parent heeft. In dat geval zetten we
\begin{equation*}
  \pi = \nil.
\end{equation*}
We zeggen dat $\pi$ \emph{niet bestaat} of \emph{niks} is. Vandaar dat we het symbool $\nil$ toevoegen aan $\sL_s$.

\section{Objecten en prototype overerving}

In §\ref{sec:prototypen} hebben we een beeld gekregen van prototype overerving. Net als scopes en omgevende scopes, blijken objecten en prototypen te modelleren met een boomstructuur. Geheel in lijn met scopes is een object een paar met daarin zijn eigen bindingen $b$ en een verwijzing naar zijn prototype $\pi$. Natuurlijk kan een object ook geen prototype hebben. Dit geven we weer aan met $\nil$. Een object $o$ is dan een element uit
\begin{equation*}
  \sO \DEF \sB \times (\sL \cup \{\nil\}).
\end{equation*}
Hierbij zijn $\sB$ weer de bindingen uit §\ref{sec:bindingen} en $\sL_o$ zijn locaties van objecten. We maken dus een strikte scheiding tussen locaties van scopes en locaties van objecten.

\section{Waarden: referenties en primitieven}
\label{sec:waarden}

[Ze worden op dezelfde manier behandeld: objecten by-reference, dus de references zelf by-value, net als primitieven -- vandaar dat ze in dezelfde verzameling waarden zitten.]

\subsection{Natuurlijke getallen}

\subsection{Functies}

\subsection{Objecten}

\section{Locaties en geheugen}\label{sec:locaties}

\section*{Extra}

\begin{align*}
  \tag*{locaties van scopes en objecten}
  \sL &\DEF \{ (n, n) \in \sN^2 \} \\
  \tag*{functies}
  \sF &\DEF \Stm \times \Id_{\langle\rangle} \times (\Id \cup \{\nil\}) \times \sL \\
  \tag*{waarden}
  \sV &\DEF \sL \cup \sN \cup \sF \\
  \tag*{binding-verzamelingen}
  \sB &\DEF \FiniteFunctions{\sV}{\Id} \\
  \tag*{objecten}
  \sO &\DEF \sB \times (\sL \cup \{\nil\}) \\
  \tag*{scopes}
  \sS &\DEF \sB \times (\sL \cup \{\nil\}) \\
\end{align*}

% vim: spell spl=nl cole=2

\chapter{Natuurlijke Semantiek}

\section{Expressies}

Expressies zijn syntactische elementen die zonder de \emph{state} van het programma te wijzigen \emph{geëvalueerd} kunnen worden tot een waarde $v \in \VV$. Voor boolse expressies geldt hetzelfde, maar deze evalueren tot de verzameling $\{\bT, \bF\}$. Specifiek getallen beschouwen we voor het gemak ook apart. Voor elk van deze drie soorten expressies definiëren we semantische functies die deze evaluatie formeel maken. Ook definiëren we een \emph{uitgebreide} semantische functie voor syntactische elementen uit $\MaybeExprs$. De signaturen van de vier functies zijn:
%
\begin{gather*}
  \Express[B]{\;}{} : \BExpr \times \MMs \times \MMo \times \LLs \times \LLo \to \{ \bT, \bF \} \\
  \Express[Z]{\;}{} : \Num \to \ZZ \\
  \Express{\;}{} : \Expr \times \MMs \times \MMo \times \LLs \times \LLo \to \VV \\
  \Express[\scalebox{1.3}{$+$}]{\;}{} : \MaybeExprs \times \MMs \times \MMo \times \LLs \times \LLo \to \VV_{\langle\rangle}
\end{gather*}

\subsection{Boolse expressies}

Er vanuit gaande dat we gewone expressie evaluatie al hebben gedefiniëerd, is de definitie van boolse expressie evaluatie heel simpel:
%
\begin{equation}
  \Express[B]{\TRUE}{\ms,\mo,\sigma,\tau} = \bT
  \tag*{[true]}
\end{equation}
%
\begin{equation}
  \Express[B]{\FALSE}{\ms,\mo,\sigma,\tau} = \bF
  \tag*{[false]}
\end{equation}
%
\begin{equation}
  \Express[B]{b_1 \AND b_2}{\ms,\mo,\sigma,\tau}
  = \begin{cases}
    \bT & \textbf{if } \Express[B]{b_1}{\ms,\mo,\sigma,\tau} = \Express[B]{b_2}{\ms,\mo,\sigma,\tau} = \bT \\
    \bF & \textbf{anders}
  \end{cases}
  \tag*{[and]}
\end{equation}
%
\begin{equation}
  \Express[B]{b_1 \OR b_2}{\ms,\mo,\sigma,\tau}
  = \begin{cases}
    \bF & \textbf{if } \Express[B]{b_1}{\ms,\mo,\sigma,\tau} = \Express[B]{b_2}{\ms,\mo,\sigma,\tau} = \bF \\
    \bT & \textbf{anders}
  \end{cases}
  \tag*{[or]}
\end{equation}
%
\begin{equation}
  \Express[B]{\NOT b}{\ms,\mo,\sigma,\tau}
  = \begin{cases}
    \bT & \textbf{if } \Express[B]{b}{\ms,\mo,\sigma,\tau} = \bF \\
    \bF & \textbf{anders}
  \end{cases}
  \tag*{[not]}
\end{equation}
%
\begin{gather*}
  \Express[B]{e_1 \sim e_2}{\ms,\mo,\sigma,\tau} =
  n_1 \sim n_2
  \tag*{[relation]} \\
  \Desdas{
    \Desda[waarbij]{n_1 = \Express{e_1}{\ms,\mo,\sigma,\tau} \in \ZZ}
    \Desda{n_2 = \Express{e_2}{\ms,\mo,\sigma,\tau} \in \ZZ}
    \Desda{(\sim) \textbf{ één van: } =, \neq, <, \le, >, \ge}
  }
\end{gather*}

\subsection{Getal expressies}

Ook evaluatie van getallen is eenvoudig:
%
\begin{align*}
  \Express[Z]{0}{} &= 0 &
  \Express[Z]{n 0}{} &= 2 \cdot \Express[N]{n}{} \\
  \Express[Z]{1}{} &= 1 &
  \Express[Z]{n 1}{} &= 2 \cdot \Express[N]{n}{} + 1
\end{align*}

\subsection{Gewone expressies}
%
\begin{equation}
  \Express{n}{\ms,\mo,\sigma,\tau} = \Express[Z]{n}{}
  \tag*{[num]}
\end{equation}
%
\begin{equation}
  \Express{\FUN (i^*) \{S\}}{\ms,\mo,\sigma,\tau} = (S, \textsc{Ids}(i^*), \nil, \sigma)
  \tag*{[function]}
\end{equation}
%
\begin{equation}
  \Express{\FUN (i^*) \RETURNS j \{S\}}{\ms,\mo,\sigma,\tau} = (S, \textsc{Ids}(i^*), j, \sigma)
  \tag*{[function w/ return]}
\end{equation}
%
\begin{equation}
  \Express{\THIS}{\ms,\mo,\sigma,\tau} = \tau
  \tag*{[this]}
\end{equation}
%
\begin{equation}
  \Express{i}{\ms,\mo,\sigma,\tau} = \big( \Finds(\ms, \sigma, i) \big)(i)
  \tag*{[identifier]}
\end{equation}
%
\begin{gather*}
  \Express{\THIS.p}{\ms,\mo,\sigma,\tau} = v
  \tag*{[this.path]} \\
  \Desdas{
    \Desda[desda]{\Trav(\mo, \tau, p) = (\omega \in \LLo, j)}
    \Desda{\Findp(\omega,j) = \omega'}
    \Desda{\mo(\omega')(j) = v}
  }
\end{gather*}
%
\begin{gather*}
  \Express{i.p}{\ms,\mo,\sigma,\tau} = v
  \tag*{[path]} \\
  \Desdas{
    \Desda[desda]{\Express{i}{\ms,\mo,\sigma,\tau} = \omega \in \LLo}
    \Desda{\Trav(\mo, \omega, p) = (\omega' \in \LLo, j)}
    \Desda{\Findp(\omega',j) = \omega''}
    \Desda{\mo(\omega'')(j) = v}
  }
\end{gather*}
%
\begin{gather*}
  \Express{e_1 \circ e_2}{\ms,\mo,\sigma,\tau} = n_1 \circ n_2
  \tag*{[op]} \\
  \Desdas{
    \Desda[desda]{n_1 = \Express{e_1}{\ms,\mo,\sigma,\tau} \in \ZZ}
    \Desda{n_2 = \Express{e_2}{\ms,\mo,\sigma,\tau} \in \ZZ}
    \Desda{(\circ) \textbf{ één van: } +, -, \times}
  }
\end{gather*}

\section{Statements}

\subsection{Basis}

Laten we beginnen met de simpelste constructie in onze taal, het lege statement $\SKIP$. Deze heeft de vorm van een axioma.
%
\begin{equation*}
  \AxiomC{$
    \Config{\SKIP}{\ms, \mo, \sigma, \tau}
    \longrightarrow
    (\ms, \mo)
  $}
  {\DisplayProof}
  \tag*{[skip]}
\end{equation*}
%
Zoals we kunnen zien zijn onze uitspraken van de vorm
%
\begin{equation*}
  \Config{S}{\ms,\mo,\sigma,\tau} \longrightarrow (\ms',\mo').
\end{equation*}
%
Hiermee bedoelen we dat
%
\begin{equation*}
  \big(\,(S,\ms,\mo,\sigma,\tau), (\ms',\mo')\,\big) \in (\longrightarrow),
\end{equation*}
%
waarbij $\longrightarrow$ de volgende signatuur heeft 
%
\begin{equation*}
  (\longrightarrow) \subseteq (\Stm \times \MMs \times \MMo \times \LLs \times \LLo) \times (\MMs \times \MMo).
\end{equation*}
%
Deze transitie werkt op een statement $S\in\Stm$ in een toestand $(\ms,\mo)\in(\MMs,\MMo)$ met als extra informatie de locatie van de huidige scope $\sigma\in\LLs$ en de locatie van het huidige $\THIS$-object $\tau\in\MMo$. Het resultaat is een nieuwe toestand in de vorm van de twee geheugens $(\ms',\mo')\in(\MMs,\MMo)$. $\SKIP$ verandert niets aan de toestand zodat $(\ms',\mo')=(\ms,\mo)$.

Voor het samenstellen van statements hebben we een regel nodig.
%
\begin{gather*}
  \AxiomC{$
    \Config{S_1}{\ms, \mo, \sigma, \tau}
    \longrightarrow
    (\ms', \mo')
  $}
  \AxiomC{$
    \Config{S_2}{\ms', \mo', \sigma, \tau}
    \longrightarrow
    (\ms'', \mo'')
  $}
  \BinaryInfC{$
    \Config{S_1; S_2}{\ms, \mo, \sigma, \tau}
    \longrightarrow
    (\ms'', \mo'')
  $}
  {\DisplayProof}
  \tag*{[comp]}
\end{gather*}
%
In dit geval geven we aan dat, wanneer we een een compositie hebben van de statements $S_1$ en $S_2$, we eerst $S_1$ uitvoeren% T: Dit is niet het juiste woord, weten even geen betere...
en daarna $S_2$. Tijdens dit proces ontstaan nieuwe toestanden, waar we natuurlijk rekening mee moeten houden. De geheugens worden dan ook netjes doorgesluisd.

Voor de controlestructuur $\IF$ hebben we twee regels nodig. De eerste is voor het geval dat de \BExpr\ evalueert in \bT, dan moet namelijk het statement van het $\THEN$-deel worden uitgevoerd. Wanneer de \BExpr\ evalueert in \bF\ moet het $\ELSE$-deel worden uitgevoerd.
Er moet dus aan een extra voorwaarde worden voldaan om deze regels toe te mogen passen. Dit zal vaker voorkomen bij de komende deductieregels. We noteren deze extra voorwaarden onder de regel of het axioma.
%Zoals vele axioma's en deductieregels heeft dit axioma een aantal voorwaarden waaraan voldaan moet worden. Deze staan eronder genoteerd, elk op een regel.  
%
\begin{gather*}
  \AxiomC{$
    \Config{S_1}{\ms, \mo, \sigma, \tau}
    \longrightarrow
    (\ms', \mo')
  $}
  \UnaryInfC{$
    \Config{\IF b \THEN  S_1 \ELSE S_2 }{\ms, \mo, \sigma, \tau}
    \longrightarrow
    (\ms', \mo')
  $}
  {\DisplayProof}
  \tag*{[if true]} \\
  \Desdas{
    \Desda[desda]{\Surr{b}^\text{B}_{\ms, \mo,\sigma,\tau} = \bT}
  }
\end{gather*}
%
en:
%
\begin{gather*}
  \AxiomC{$
    \Config{S_2}{\ms, \mo, \sigma, \tau}
    \longrightarrow
    (\ms', \mo')
  $}
  \UnaryInfC{$
    \Config{\IF b \THEN S_1 \ELSE S_2 }{\ms, \mo, \sigma, \tau}
    \longrightarrow
    (\ms', \mo')
  $}
  {\DisplayProof}
  \tag*{[if false]} \\
  \Desdas{
    \Desda{\Surr{b}^\text{B}_{\ms, \mo,\sigma,\tau} = \bF}
  }
\end{gather*}
%
Eenzelfde tactiek passen we toe bij een $\WHILE$-loop.
%
\begin{gather*}
  \AxiomC{$
    \Config{S_1}{\ms, \mo, \sigma, \tau}
    \longrightarrow
    (\ms',\mo')
  $}
  \AxiomC{$
    \Config{\WHILE b \DO S_1 }{\ms', \mo', \sigma, \tau}
    \longrightarrow
    (\ms'',\mo'')
  $}
  \BinaryInfC{$
    \Config{\WHILE b \DO S_1 }{\ms, \mo, \sigma, \tau}
    \longrightarrow
    (\ms'',\mo'')
  $}
  {\DisplayProof}
  \tag*{[while true]} \\
  \Desdas{
    \Desda{\Surr{b}^\text{B}_{m,\sigma,\tau} = \bT}
  }
\end{gather*}
%
en:
%
\begin{gather*}
  \AxiomC{$
    \Config{\WHILE b \DO S_1 }{\ms, \mo, \sigma, \tau}
    \longrightarrow
    (\ms, \mo)
  $}
  {\DisplayProof}
  \tag*{[while false]} \\
  \Desdas{
    \Desda{\Surr{b}^\text{B}_{\ms,\mo,\sigma,\tau} = \bF}
  }
\end{gather*}

\subsection{Variabelen}

We komen nu bij een interessanter deel van de taal, namelijk het \emph{declareren} van variabelen en het \emph{toekennen} van waarden. In deze sectie gaan we dus alleen de bereiken modificeren. Aan de basis hiervan ligt het declareren van een variabele met $\LOCAL$. We willen in het huidige bereik de eigen bindingen zó aanpassen, dat de \Id\ die we declareren gekoppeld wordt aan $\nil$ (de variabele bevat immers nog geen expliciete waarde). Voordat we dit kunnen doen, moeten we bereik $\sigma$ opzoeken in het geheugen voor bereiken $\ms$ met $\ms(\sigma)$. Vervolgens passen we de binding hiervan aan (zie notatie in §\ref{sec:bereiken} en §\ref{sec:uitbreiden}). Met de verwijzing naar het omliggend bereik $\pi$ doen we niks. Dit paar van uitgebreide bindingen en oud omliggend bereik zetten we terug in het geheugen $\ms$ op plek $\sigma$.
%
\begin{gather*}
  \AxiomC{$
    \Config{\LOCAL i}{\ms, \mo, \sigma, \tau}
    \longrightarrow
    (\ms', \mo)
  $}
  {\DisplayProof}
  \tag*{[local]} \\
  \Desdas{
    \Desda{\ms' = \ms \surr{ \sigma \mapsto \big(b_{\ms(\sigma)}[i \mapsto \BOT], \pi_{\ms(\sigma)}\big) }}
  }
\end{gather*}
%
Wanneer we daadwerkelijk een waarde aan een variabele willen toekennen, doen we dit met dezelfde aanpassingstechniek als hierboven. Er is echter één groot verschil. Voordat we een waarde aan een variabele kunnen koppelen, moeten we eerst het bereik vinden waarin deze gedeclareerd is. Dit hoeft niet het huidige bereik te zijn en dus zoeken we deze op met de hulpfunctie $\Finds$. Daarnaast moeten we natuurlijk de expressie aan de rechter kant van het is-teken evalueren.
%
\begin{gather*}
  \AxiomC{$
    \Config{i = e}{\ms, \mo, \sigma, \tau}
    \longrightarrow
    (\ms', \mo)
  $}
  {\DisplayProof}
  \tag*{[assign identifier]} \\
  \Desdas{
    \Desda{\sigma_\text{def} = \Finds(\ms, \sigma, i)}
    \Desda{\Surr{e}_{\ms, \mo, \sigma, \tau} = v}
    \Desda{\ms'= \ms \surr{ \sigma_\text{def} \mapsto \big(b_{\ms(\sigma_\text{def})}[ i \mapsto v ], \pi_{\ms(\sigma_\text{def})}\big) }}
  }
\end{gather*}

\subsection{Objecten}

Bij attributen gaat toekenning net iets anders. We hebben de hulp nodig van andere functies en we moeten rekening houden met het doorlopen van een pad. Daarnaast is er nog het speciale geval dat het eerste deel van het pad $\THIS$ is. Natuurlijk gebeuren al deze aanpassingen in het geheugen voor objecten $\mo$ en gebruiken we locaties van objecten $\omega\in\LLo$ in plaats van locaties van bereiken.
%
\begin{gather*}
  \AxiomC{$
    \Config{\THIS.s = e}{\ms, \mo, \sigma, \tau}
    \longrightarrow
    (\ms, \mo')
  $}
  {\DisplayProof}
  \tag*{[assign this.path]} \\
  \Desdas{
    \Desda{\Trav(\mo, \tau, s) = (\omega, i)}
    \Desda{\Surr{e}_{\ms, \mo, \sigma, \tau} = v}
    \Desda{\mo'= \mo \surr{ \omega \mapsto \big(b_{\mo(\omega)}[ i \mapsto v ], \pi_{\mo(\omega)}\big) }}
  }
\end{gather*}
%
\begin{gather*}
  \AxiomC{$
    \Config{i.s = e}{\ms, \mo, \sigma, \tau}
    \longrightarrow
    (\ms', \mo)
  $}
  {\DisplayProof}
  \tag*{[assign path]} \\
  \Desdas{
    \Desda{\sigma_\text{def} = \Finds(\ms, \sigma, i)}
    \Desda{b_{\ms(\sigma_\text{def})}(i) = \omega \in \LL}
    \Desda{\Trav(\mo, \omega, s) = (\omega', j)}
    \Desda{\Surr{e}_{\ms, \mo, \sigma, \tau} = v}
    \Desda{\ms'= \ms \surr{ \omega' \mapsto \big(b_{\ms(\omega')}[ j \mapsto v ], \pi_{\ms(\omega')}\big) }}
  }
\end{gather*}
%
Bij attributen gaat toekenning net iets anders. We hebben de hulp nodig van andere functies en we moeten rekening houden met het doorlopen van een pad. Daarnaast is er nog het speciale geval dat het eerste deel van het pad $\THIS$ is. Natuurlijk gebeuren al deze aanpassingen in het geheugen voor objecten $\mo$ en gebruiken we locaties van objecten $\omega\in\LLo$ in plaats van locaties van bereiken.
%
\begin{gather*}
  \AxiomC{$
    \Config{\THIS.s = e}{\ms, \mo, \sigma, \tau}
    \longrightarrow
    (\ms, \mo')
  $}
  {\DisplayProof}
  \tag*{[assign this.path]} \\
  \Desdas{
    \Desda{\Trav(\mo, \tau, s) = (\omega, i)}
    \Desda{\Surr{e}_{\ms, \mo, \sigma, \tau} = v}
    \Desda{\mo'= \mo \surr{ \omega \mapsto \big(b_{\mo(\omega)}[ i \mapsto v ], \pi_{\mo(\omega)}\big) }}
  }
\end{gather*}
%
\begin{gather*}
  \AxiomC{$
    \Config{i.s = e}{\ms, \mo, \sigma, \tau}
    \longrightarrow
    (\ms', \mo)
  $}
  {\DisplayProof}
  \tag*{[assign path]} \\
  \Desdas{
    \Desda{\sigma_\text{def} = \Finds(\ms, \sigma, i)}
    \Desda{b_{\ms(\sigma_\text{def})}(i) = \omega \in \LL}
    \Desda{\Trav(\mo, \omega, s) = (\omega', j)}
    \Desda{\Surr{e}_{\ms, \mo, \sigma, \tau} = v}
    \Desda{\ms'= \ms \surr{ \omega' \mapsto \big(b_{\ms(\omega')}[ j \mapsto v ], \pi_{\ms(\omega')}\big) }}
  }
\end{gather*}
%
Nu hebben we nog niet bekeken hoe we aangeven dat een variabele een object bevat. In §\ref{sec:waarden} is uitgebreid besproken dat er een significant verschil is tussen primitieve waarden en objecten. Toch gaat dit op bijna dezelfde manier als het toekennen van een primitieve waarde. De locatie $\omega$ van een object $o$ kan immers wel op dezelfde manier worden behandeld. Wat wel moet gebeuren is het aanmaken van een nieuw, leeg object in het geheugen voor objecten. Zo een leeg object heeft de vorm $(\varnothing, \nil)$. De bindingen zijn immers leeg en er is nog geen prototype gedefinieerd.
%
\begin{gather*}
  \AxiomC{$
    \Config{i \OBJECT}{\ms, \mo, \sigma, \tau}
    \longrightarrow
    (\ms', \mo')
  $}
  {\DisplayProof}
  \tag*{[object]} \\
  \Desdas{
    \Desda{\Finds(\ms, \sigma, i) = \sigma_\text{def}}
    \Desda{\omega = \Nexto(\mo)}
    \Desda{\ms' = \ms \surr{ \sigma_\text{def} \mapsto \big(b_{\ms(\sigma')}[i\mapsto \omega], \pi_{\ms(\sigma')}\big) }}
    \Desda{\mo' = \mo \surr{ \omega \mapsto \big(\varnothing, \BOT\big) }}
  }
\end{gather*}
%
Het definiëren van een prototype gaat met het $\CLONES$-statement. Hiervoor zoeken we simpelweg de locaties $\omega_i$ en $\omega_j$ van de twee objecten op. Vervolgens zetten we in het geheugen dat het prototype van het object in $\omega_i$ de locatie $\omega_j$ is.
%
\begin{gather*}
  \AxiomC{$
    \Config{i \CLONES j}{\ms, \mo, \sigma, \tau}
    \longrightarrow
    (\ms, \mo')
  $}
  {\DisplayProof}
  \tag*{[clones]} \\
  \Desdas{
    \Desda{\Surr{i}_{\ms,\mo,\sigma,\tau} = \omega_i \in \LL}
    \Desda{\Surr{j}_{\ms,\mo,\sigma,\tau} = \omega_j \in \LL}
    \Desda{\mo' = \mo \surr{ \omega_i \mapsto \big(b_{\mo(\omega_i)}, \omega_j\big) }}
  }
\end{gather*}

\subsection{Functies}
%
\dots Komt nog \dots
%
\begin{gather*}
  \AxiomC{$
    \Config{S_{\!f}}{\ms', \mo, \sigma_{\!f\text{new}}, \omega'}
    \longrightarrow
    (\ms'', \mo'')
  $}
  \UnaryInfC{$
    \Config{i.s(e^*)}{\ms, \mo, \sigma, \tau}
    \longrightarrow
    (\ms'', \mo'')
  $}
  {\DisplayProof}
  \tag*{[call]} \\
  \Desdas{
    \Desda{\sigma_\text{def} = \Finds(\ms, \sigma, i)}
    \Desda{b_{\ms(\sigma_\text{def})}(i) = \omega \in \LL}
    \Desda{\Trav(\mo, \omega, s) = (\omega', j)}
    \Desda{(S_{\!f}, I_{\!f}, i_{\!f}, \sigma_{\!f\text{def}}) = f = b_{\mo(\omega')}(j)}
    \Desda{\sigma_{\!f\text{new}} = \Nexts(\ms)}
    \Desda{\ms' = \ms\surr{ \sigma_{\!f\text{new}} \mapsto \big(\Surr{e^*}^*_{\ms,\sigma,\tau}(I_{\!f}), \sigma_{\!f\text{def}}\big) }}
  }
\end{gather*}

%\section*{Extra}

%Deze teksten zijn vooral bedoeld als “tekstvlees” (lorem ipsum's). We zullen axioma's en deductieregels introduceren waarmee we de relatie $(\longrightarrow)$ definiëren, die de volgende signatuur heeft:

%$$ (\longrightarrow) \subseteq (\Stm \times \MM \times \LL \times \LL) \times \MM $$

%Wanneer we een uitspraak doen van de vorm:

%$$ \Config{S}{m,\sigma,\tau} \longrightarrow m' $$

%..dan bedoelen we daarmee dat:

%$$ \big(\,(S,m,\sigma,\tau), m'\,\big) \in (\longrightarrow) $$

%Deze uitspraak moet je lezen als: “In de toestand met geheugen $m$, scope $\sigma$ en \emph{this} object $\tau$, termineert het statement $S$, waarbij het resultaat-geheugen $m'$ is.”
%
%Een van deze axioma's [object], heeft betrekking tot de productieregel in de grammatica die de $\OBJECT$ “literal” introduceert.
%
%Wanneer bij een dergelijke opsomming van voorwaarden een nieuwe variabele wordt geïntroduceerd zoals hierboven, met de volgende vorm: $\textbf{desda } \square = \theta \dots$; dan moet deze gelezen worden als: $\textbf{desda } \exists_\theta \surr{ \square = \theta \dots }$.

% vim: syn=latex spell spl=nl cole=1

% !TeX root = werkstuk.tex
% !TeX program = xelatex

\chapter{Case study}

In de case study maken we aannemelijk dat de natuurlijke semantiek daadwerkelijk de werking beschrijft die wij voor ogen hebben. Dit doen we door een volledige afleidingsboom te geven voor de executie het volgende exemplarische programma, vanuit een speciale \emph{start-toestand} $\State{\phantom{A}} \DEF \langle(\varnothing, \nil)\rangle, \langle\rangle$:

\newCodeFragment
\begin{minipage}{.5\textwidth}
\codeLines{
  \codeLine{\LOCAL \id{f}}
  \codeLine{\id{f} = \FUN(\id{n}) \RETURNS \id{g}}
  \codeLine{\IN \LOCAL \id{g}}
  \codeLine{\IN \id{g} = \FUN(\;) \RETURNS \id{n}}
  \codeLine{\IN \IN \id{n} = \id{n} + 1}
}
\end{minipage}
\begin{minipage}{.5\textwidth}
\codeLines{
  \codeLine{\LOCAL \id{c}}
  \codeLine{\id{c} = \id{f}(5)}
  \codeLine{\id{c}(\;)}
  \codeLine{\LOCAL \id{i}}
  \codeLine{\id{i} = \id{c}(\;)}
}
\end{minipage}

Deze afleiding is op de volgende twee bladzijden in delen weergegeven.

%
%\begin{gather}
%  b_{\ms(@0)}(i) = 7
%  \textbf{~~waarbij~~}
%  \vTerminates{
%    \LOCAL \id{f} \\
%    \id{f} = \FUN(\id{n}) \RETURNS \id{g} \\
%    \IN \LOCAL \id{g} \\
%    \IN \id{g} = \FUN(\;) \RETURNS \id{n} \\
%    \IN \IN \id{n} = \id{n} + 1 \\
%    \LOCAL \id{c} \\
%    \id{c} = \id{f}(5) \\
%    \id{c}(\;) \\
%    \LOCAL \id{i} \\
%    \id{i} = \id{c}(\;)
%  }
%    {\State{\phantom{A}}}
%    {@0, \nil}
%    {\ms, \mo}
%  \label{cA} \\
%  \Express{\id{y}.\id{a}}{\ms, \mo, @0, \nil} = 42
%  \textbf{~~waarbij~~}
%  \vTerminates{
%    \LOCAL \id{x} \\
%    \id{x} \OBJECT \\
%    \id{x}.\id{a} = 42 \\
%    \LOCAL \id{y} \\
%    \id{y} \OBJECT \\
%    \id{y} \CLONES \id{x}
%  }
%    {\State{\phantom{A}}}
%    {@0, \nil}
%    {\ms, \mo}
%  \label{cB}
%\end{gather}
%
%Dit soort bewijzen gaan niet al te makkelijk met een natuurlijke semantiek, maar het is buiten de focus van dit werkstuk om een bijpassend axiomatische semantiek te ontwikkelen.
%
%We zullen daarom een groot deel van het bewijs dat redelijk simpel van karakter is (herhaaldelijk [comp], [local] etc) weglaten, en vooral de essentiële stappen bewijzen.
%
%Het volgende voorbeeld geeft weer hoe men bij simpele afleidingen met [comp], [local] en [assign var] te werk gaat. Laat $\State{A}$ en $\State{B}$ als volgt gedefinieerd zijn:
%%
%\begin{align*}
%f &\DEF \big<\left[\SmallProgram{
%        \LOCAL \id{g} \\
%        \id{g} = \FUN(\;) \RETURNS \id{n} \\
%        \IN \id{n} = \id{n} + 1
%      }\right], \langle\id{n}\rangle, \id{g}, @0\big> \\
%\State{A} &\DEF \langle([\id{f} \mapsto \nil], \nil)\rangle, \langle\rangle \\
%\State{B} &\DEF \langle([\id{f} \mapsto f], \nil)\rangle, \langle\rangle
%\end{align*}
%%
%Het eerste fragment voor het bewijs voor (\ref{cA}) gaat dan als volgt:
%%
%\begin{equation*}
%\AxiomC{}
%\LeftLabel{[local]}
%\UnaryInfC{$
%  \Terminates{\LOCAL \id{f}}
%    {\State{\phantom{A}}}
%    {@0, \nil}
%    {\State{A}}
%$}
%\AxiomC{}
%\LeftLabel{[assign var]}
%\UnaryInfC{$
%  \vTerminates{
%    \id{f} = \FUN(\id{n}) \RETURNS \id{g} \\
%    \IN \LOCAL \id{g} \\
%    \IN \id{g} = \FUN(\;) \RETURNS \id{n} \\
%    \IN \IN \id{n} = \id{n} + 1
%  }
%    {\State{A}}
%    {@0, \nil}
%    {\State{B}}
%$}
%\LeftLabel{[comp]}
%\BinaryInfC{$
%  \vTerminates{
%    \LOCAL \id{f} \\
%    \id{f} = \FUN(\id{n}) \RETURNS \id{g} \\
%    \IN \LOCAL \id{g} \\
%    \IN \id{g} = \FUN(\;) \RETURNS \id{n} \\
%    \IN \IN \id{n} = \id{n} + 1
%  }
%    {\State{\phantom{A}}}
%    {@0, \nil}
%    {\State{B}}
%$}
%\DisplayProof{}
%\end{equation*}
%%
%Nu nemen we aan dat bewezen is dat:
%%
%\begin{equation*}
%  \vTerminates{
%    \LOCAL \id{f} \\
%    \id{f} = \FUN(\id{n}) \RETURNS \id{g} \\
%    \IN \LOCAL \id{g} \\
%    \IN \id{g} = \FUN(\;) \RETURNS \id{n} \\
%    \IN \IN \id{n} = \id{n} + 1 \\
%    \LOCAL \id{c}
%  }
%    {\State{\phantom{A}}}
%    {@0, \nil}
%    {\State{C}}
%  \textbf{~~waarbij~~}
%  \State{C} \DEF \langle(\left[\Bindings{
%    \id{f} & f \\
%    \id{c} & \nil
%  }\right], \nil)\rangle, \langle\rangle
%\end{equation*}
%%
%De volgende 

\renewcommand{\SmallProgramSize}{.75}


\def\fgABCDetc{\begin{align*}
f &\DEF \big<\left[\SmallProgram{
    \LOCAL \id{g} \\
    \id{g} = \FUN(\;) \RETURNS \id{n} \\
    \IN \id{n} = \id{n} + 1
  }\right], \langle\id{n}\rangle, \id{g}, @0\big> &
\State{D} &\DEF \langle(\left[\Bindings{
    \id{f} & f \\
    \id{c} & \nil
  }\right], \nil), ([\id{n} \mapsto 5], @0)\rangle, \langle\rangle &
\State{I} &\DEF \langle(\left[\Bindings{
    \id{f} & f \\
    \id{c} & g
  }\right], \nil), (\left[\Bindings{
    \id{n} & 6 \\
    \id{g} & g
  }\right], @0), (\varnothing, @1)\rangle, \langle\rangle \\
g &\DEF \big<[\id{n} = \id{n} + 1], \langle\rangle, \id{n}, @1\big> &
\State{E} &\DEF \langle(\left[\Bindings{
    \id{f} & f \\
    \id{c} & \nil
  }\right], \nil), (\left[\Bindings{
    \id{n} & 5 \\
    \id{g} & \nil
  }\right], @0)\rangle, \langle\rangle &
\State{J} &\DEF \langle(\left[\Bindings{
    \id{f} & f \\
    \id{c} & g \\
    \id{i} & \nil
  }\right], \nil), (\left[\Bindings{
    \id{n} & 6 \\
    \id{g} & g
  }\right], @0), (\varnothing, @1)\rangle, \langle\rangle \\
\State{A} &\DEF \langle([\id{f} \mapsto \nil], \nil)\rangle, \langle\rangle &
\State{F} &\DEF \langle(\left[\Bindings{
    \id{f} & f \\
    \id{c} & \nil
  }\right], \nil), (\left[\Bindings{
    \id{n} & 5 \\
    \id{g} & g
  }\right], @0)\rangle, \langle\rangle &
\State{K} &\DEF \langle(\left[\Bindings{
    \id{f} & f \\
    \id{c} & g \\
    \id{i} & \nil
  }\right], \nil), (\left[\Bindings{
    \id{n} & 6 \\
    \id{g} & g
  }\right], @0), (\varnothing, @1), (\varnothing, @1)\rangle, \langle\rangle \\
\State{B} &\DEF \langle([\id{f} \mapsto f], \nil)\rangle, \langle\rangle &
\State{G} &\DEF \langle(\left[\Bindings{
    \id{f} & f \\
    \id{c} & g
  }\right], \nil), (\left[\Bindings{
    \id{n} & 5 \\
    \id{g} & g
  }\right], @0)\rangle, \langle\rangle &
\State{L} &\DEF \langle(\left[\Bindings{
    \id{f} & f \\
    \id{c} & g \\
    \id{i} & \nil
  }\right], \nil), (\left[\Bindings{
    \id{n} & 7 \\
    \id{g} & g
  }\right], @0), (\varnothing, @1), (\varnothing, @1)\rangle, \langle\rangle \\
\State{C} &\DEF \langle(\left[\Bindings{
    \id{f} & f \\
    \id{c} & \nil
  }\right], \nil)\rangle, \langle\rangle &
\State{H} &\DEF \langle(\left[\Bindings{
    \id{f} & f \\
    \id{c} & g
  }\right], \nil), (\left[\Bindings{
    \id{n} & 5 \\
    \id{g} & g
  }\right], @0), (\varnothing, @1)\rangle, \langle\rangle &
\State{M} &\DEF \langle(\left[\Bindings{
    \id{f} & f \\
    \id{c} & g \\
    \id{i} & 7
  }\right], \nil), (\left[\Bindings{
    \id{n} & 7 \\
    \id{g} & g
  }\right], @0), (\varnothing, @1), (\varnothing, @1)\rangle, \langle\rangle
\end{align*}}

\begin{Landscape}
\begin{prooftree}

\AxiomC{}
\RuleLabel{[local]}
\UnaryInfC{$
  \Terminates{\LOCAL \id{f}}
    {\State{\phantom{A}}}
    {@0, \nil}
    {\State{A}}
$}
\AxiomC{}
\RuleLabel{[assign]}
\UnaryInfC{$
  \vTerminates{
    \id{f} = \FUN(\id{n}) \RETURNS \id{g} \\
    \IN \LOCAL \id{g} \\
    \IN \id{g} = \FUN(\;) \RETURNS \id{n} \\
    \IN \IN \id{n} = \id{n} + 1
  }
    {\State{A}}
    {@0, \nil}
    {\State{B}}
$}
\RuleLabel{[comp]}
\BinaryInfC{$
  \vTerminates{
    \LOCAL \id{f} \\
    \id{f} = \FUN(\id{n}) \RETURNS \id{g} \\
    \IN \LOCAL \id{g} \\
    \IN \id{g} = \FUN(\;) \RETURNS \id{n} \\
    \IN \IN \id{n} = \id{n} + 1
  }
    {\State{\phantom{A}}}
    {@0, \nil}
    {\State{B}}
$}
\AxiomC{}
\RuleLabel{[local]}
\UnaryInfC{$
  \Terminates{\LOCAL \id{c}}
    {\State{\phantom{B}}}
    {@0, \nil}
    {\State{C}}
$}
\RuleLabel{[comp]}
\BinaryInfC{$
  \vTerminates{
    \LOCAL \id{f} \\
    \id{f} = \FUN(\id{n}) \RETURNS \id{g} \\
    \IN \LOCAL \id{g} \\
    \IN \id{g} = \FUN(\;) \RETURNS \id{n} \\
    \IN \IN \id{n} = \id{n} + 1 \\
    \LOCAL \id{c}
  }
    {\State{\phantom{A}}}
    {@0, \nil}
    {\State{C}}
$}
\AxiomC{}
\RuleLabel{[local]}
\UnaryInfC{$
  \Terminates{\LOCAL \id{g}}
    {\State{D}}
    {@1, \nil}
    {\State{E}}
$}
\AxiomC{}
\RuleLabel{[assign]}
\UnaryInfC{$
  \vTerminates{
    \id{g} = \FUN(\;) \RETURNS \id{n} \\
    \IN \id{n} = \id{n} + 1
  }
    {\State{E}}
    {@1, \nil}
    {\State{F}}
$}
\RuleLabel{[comp]}
\BinaryInfC{$
  \vTerminates{
    \LOCAL \id{g} \\
    \id{g} = \FUN(\;) \RETURNS \id{n} \\
    \IN \id{n} = \id{n} + 1
  }
    {\State{D}}
    {@1, \nil}
    {\State{F}}
$}
\RuleLabel{[call]}
\UnaryInfC{$
  \Terminates{\id{c} = \id{f}(5)}
    {\State{C}}
    {@0, \nil}
    {\State{G}}
$}
\RuleLabel{[comp]}
\BinaryInfC{$
  \vTerminates{
    \LOCAL \id{f} \\
    \id{f} = \FUN(\id{n}) \RETURNS \id{g} \\
    \IN \LOCAL \id{g} \\
    \IN \id{g} = \FUN(\;) \RETURNS \id{n} \\
    \IN \IN \id{n} = \id{n} + 1 \\
    \LOCAL \id{c} \\
    \id{c} = \id{f}(5)
  }
    {\State{\phantom{A}}}
    {@0, \nil}
    {\State{G}}
$}
\dottedLine
\UnaryInfC{\hspace*{.8\textwidth}}
\noLine
\UnaryInfC{(afgekapt, zie volgend blad)}
\end{prooftree}
\null
\vfill
\hrule
\fgABCDetc
\clearpage
\begin{prooftree}
\AxiomC{(vervolg afleiding, zie vorig blad)}
\dottedLine
\UnaryInfC{$
  \vTerminates{
    \LOCAL \id{f} \\
    \id{f} = \FUN(\id{n}) \RETURNS \id{g} \\
    \IN \LOCAL \id{g} \\
    \IN \id{g} = \FUN(\;) \RETURNS \id{n} \\
    \IN \IN \id{n} = \id{n} + 1 \\
    \LOCAL \id{c} \\
    \id{c} = \id{f}(5)
  }
    {\State{\phantom{A}}}
    {@0, \nil}
    {\State{G}}
$}
\AxiomC{}
\RuleLabel{[assign]}
\UnaryInfC{$
  \Terminates{\id{n} = \id{n} + 1}
    {\State{H}}
    {@2, \nil}
    {\State{I}}
$}
\RuleLabel{[call]}
\UnaryInfC{$
  \Terminates{\id{c}()}
    {\State{G}}
    {@0, \nil}
    {\State{I}}
$}
\RuleLabel{[comp]}
\BinaryInfC{$
  \vTerminates{
    \LOCAL \id{f} \\
    \id{f} = \FUN(\id{n}) \RETURNS \id{g} \\
    \IN \LOCAL \id{g} \\
    \IN \id{g} = \FUN(\;) \RETURNS \id{n} \\
    \IN \IN \id{n} = \id{n} + 1 \\
    \LOCAL \id{c} \\
    \id{c} = \id{f}(5) \\
    \id{c}()
  }
    {\State{\phantom{A}}}
    {@0, \nil}
    {\State{I}}
$}
\AxiomC{}
\RuleLabel{[local]}
\UnaryInfC{$
  \Terminates{\LOCAL \id{i}}
    {\State{I}}
    {@0, \nil}
    {\State{J}}
$}
\RuleLabel{[comp]}
\BinaryInfC{$
  \vTerminates{
    \LOCAL \id{f} \\
    \id{f} = \FUN(\id{n}) \RETURNS \id{g} \\
    \IN \LOCAL \id{g} \\
    \IN \id{g} = \FUN(\;) \RETURNS \id{n} \\
    \IN \IN \id{n} = \id{n} + 1 \\
    \LOCAL \id{c} \\
    \id{c} = \id{f}(5) \\
    \id{c}() \\
    \LOCAL \id{i}
  }
    {\State{\phantom{A}}}
    {@0, \nil}
    {\State{J}}
$}
\AxiomC{}
\RuleLabel{[assign]}
\UnaryInfC{$
  \Terminates{\id{n} = \id{n} + 1}
    {\State{K}}
    {@2, \nil}
    {\State{L}}
$}
\RuleLabel{[call]}
\UnaryInfC{$
  \Terminates{\id{i} = \id{c}()}
    {\State{J}}
    {@0, \nil}
    {\State{M}}
$}
\RuleLabel{[comp]}
\BinaryInfC{$
  \vTerminates{
    \LOCAL \id{f} \\
    \id{f} = \FUN(\id{n}) \RETURNS \id{g} \\
    \IN \LOCAL \id{g} \\
    \IN \id{g} = \FUN(\;) \RETURNS \id{n} \\
    \IN \IN \id{n} = \id{n} + 1 \\
    \LOCAL \id{c} \\
    \id{c} = \id{f}(5) \\
    \id{c}() \\
    \LOCAL \id{i} \\
    \id{i} = \id{c}()
  }
    {\State{\phantom{A}}}
    {@0, \nil}
    {\State{M}}
$}
\end{prooftree}
\null
\vfill
\hrule
\fgABCDetc
\end{Landscape}


\backmatter

\end{document}

% vim: syn=latex spell spl=nl cole=1 cocu=nv
